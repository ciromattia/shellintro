\chapter{Nozioni di base}

\section{Caratteristiche della shell}
Ogni shell (non solo bash) presenta una serie di caratteristiche, raggruppabili in macrocategorie:
\begin{itemize}
	\item comandi built-in;
	\item metacaratteri;
	\item redirezione di I/O;
	\item command substitution;
	\item sequenze condizionali e non;
	\item raggruppamento di comandi;
	\item esecuzione in background;
	\item quoting;
	\item subshell;
	\item variabili (locali/d'ambiente);
	\item scripting.
\end{itemize}
Passiamo a vederli in generale.

\section{Comandi built-in}
Quando richiedete l'esecuzione di un comando esterno, la shell si
occupa di richiamare il corrispondente file eseguibile, caricarlo in memoria
ed eseguirlo (ad esempio, quando chiamiamo il comando \texttt{ls},
\textit{bash} si occupa di trovare e lanciare \texttt{/bin/ls}).

Al contrario, quando richiedete un comando che possa essere riconosciuto ed
eseguito direttamente dalla shell, fate uso di un comando \textit{interno},
detto anche \textit{built-in}.\\
Esempi:
\begin{verbatim}
  echo         # stampa tutti i suoi argomenti sullo standard output
  cd           # cambia directory
\end{verbatim}

\section{Metacaratteri}
La shell dispone di una serie di caratteri che non sono considerati nella loro
forma a video, ma servono a specificare serie di funzionalit\`a interne alla
shell. Per questo motivo quando volete digitare uno di tali caratteri dovete
solitamente farne l'\textit{escape} con il carattere \textit{backslash}
(``$\backslash$'').

I metacaratteri possono essere trattati in maniera leggermente diversa a
seconda della shell utilizzata, ma in generale sono questi\footnote{e
\textit{bash} segue questo elenco}:
\begin{itemize}
	\item \texttt{>, >>, <\ \ \ \ \ \ \ \ } redirezione I/O
	\item \texttt{|\ \ \ \ \ \ \ \ \ \ \ \ \ \ } pipe
	\item \texttt{*, ?, [\ldots]\ \ \ \ } wildcards
	\item \texttt{`command`\ \ \ \ \ \ } command substitution
	\item \texttt{;\ \ \ \ \ \ \ \ \ \ \ \ \ \ } esecuzione sequenziale
	\item \texttt{||, \&\&\ \ \ \ \ \ \ \ \ } esecuzione condizionale
	\item \texttt{(\ldots)\ \ \ \ \ \ \ \ \ \ } raggruppamento comandi
	\item \texttt{\&\ \ \ \ \ \ \ \ \ \ \ \ \ \ } esecuzione in background
	\item \texttt{"", '{}'\ \ \ \ \ \ \ \ \ \ } quoting
	\item \texttt{\#\ \ \ \ \ \ \ \ \ \ \ \ \ \ } commento
	\item \texttt{\$\ \ \ \ \ \ \ \ \ \ \ \ \ \ } espansione di variabile
	\item \texttt{$\backslash$\ \ \ \ \ \ \ \ \ \ \ \ \ \ } carattere di backslash
	\item \texttt{<<\ \ \ \ \ \ \ \ \ \ \ \ \ \ } ``here documents''
\end{itemize}

\section{Redirezione I/O}
Ogni processo \`e associato a tre ``stream'': lo \textit{standard input}
(\texttt{stdin}), lo \textit{standard output} (\texttt{stdout}) e lo
\textit{standard error} (\texttt{stderr}).

La redirezione dell'I/O permette di associare questi stream a
sorgenti/destinazioni\footnote{ovviamente nel caso di \texttt{stdin} avremo
una sorgente, mentre per \texttt{stdout} e \texttt{stderr} avremo due
destinazioni -- che \emph{possono} coincidere} diverse dalle loro attuali.
Possiamo in questo modo salvare l'output di un processo in un file di testo,
oppure possiamo salvare il suo log di errore, attraverso semplici comandi si
questo tipo:
\begin{verbatim}
  ls > list.txt     # salva l'output di ls in un file list.txt
  ls >> list.txt    # aggiunge l'output di ls in coda
                    # al file list.txt
  rm fileinutile >& /dev/null  # redireziona stdout e stderr
                               # di rm su /dev/null
\end{verbatim}
Notate che, se non specificato, ``>'' ridirige lo \texttt{stdout}, e ``<'' lo
\texttt{stdin}. ``>\&'' ridirige \emph{tutti} i flussi di output (quindi anche
lo \texttt{stderr}). Notate anche il diverso significato di ``>'' e ``>>'': il
primo se rediretto in un file lo tronca al suo inizio, cancellando il suo
contenuto prima di scrivere il flusso, mentre il secondo conserva il contenuto
del file e accoda ad esso il flusso.

\addcontentsline{toc}{subsection}{Pipe, o catene di montaggio}
\subsection*{Pipe, o catene di montaggio}
Attraverso le pipe, potete utilizzare l'output di un processo come input di un
altro processo:
\begin{verbatim}
$ ls
  a.c b.c a.o b.o dir1 dir2
$ ls | wc -w
  6
\end{verbatim}
\texttt{wc} (un programma che conta parole, linee e molto altro) prende come
\texttt{stdin} lo \texttt{stdout} di \texttt{ls}, ritornando il numero di
parole listate da quest'ultimo: 6, appunto.

\medskip
Potrebbe sembrare una grossa limitazione poter unicamente ridirigere il
flusso, e non poterlo duplicare: a questo supplisce il comando \texttt{tee}
\footnote{il nome deriva dai ``giunti a T'', utilizzati dagli idraulici}
che copia lo \texttt{stdin} sul file specificato E sullo \texttt{stdout}:
\begin{verbatim}
$ who | tee who_capture.txt | sort
\end{verbatim}
Se \texttt{tee} viene chiamato con l'opzione \texttt{-a}, viene fatto l'append
al file argomento, invece del troncamento.

\section{Wildcards}
Le wildcards sono utilizzate per specificare pattern comprendenti pi\`u file:
i file vengono passati in rassegna, e quelli che corrispondono\footnote{in
gergo, ``\emph{to match}''} vengono restituiti. Vengono utlizzati pi\`u spesso
delle \textit{regular expressions} per la loro semplicit\`a, ad ogni modo sono
un sottoinsieme di queste ultime.
\begin{itemize}
	\item \texttt{*\ \ \ \ \ } match di qualsiasi stringa (anche vuota)
	\item \texttt{?\ \ \ \ \ } match di qualsiasi carattere singolo
	\item \texttt{[\ldots]\ } match di qualsiasi carattere inserito nelle
		parentesi
\end{itemize}
Esempi:
\begin{verbatim}
$ ls *.c
 prova001.c prova004.c prova125.c prov.c
$ ls prova00?.c
 prova001.c prova004.c
$ ls prova[0-9][0-9][0-9].c
 prova001.c prova004.c prova125.c
\end{verbatim}

\section{Command substitution}
Si pu\`o sostituire ad un argomento un comando del tipo
\texttt{`comando`}: il suo standard output verr\`a sostituito al comando
stesso:
\begin{verbatim}
$ echo Data odierna: `date`
$ echo Utenti collegati: `who | wc -l`
$ tar czvf src-`date`.tar.gz src/
\end{verbatim}

\section{Sequenze}
\addcontentsline{toc}{subsection}{Non condizionali}
\subsection*{Non condizionali}
Si possono eseguire sequenzialmente dei comandi utilizzando il metacarattere
``;'':
\begin{verbatim}
$ date ; pwd ; ls
\end{verbatim}

\addcontentsline{toc}{subsection}{Condizionali}
\subsection*{Condizionali}
Le sequenze condizionali specificano la sequenza da eseguire in base
all'\emph{exit code} del comando precedente all'interno della sequenza.

``||'' viene utilizzato per eseguire il comando nel caso l'\emph{exit code}
del precedente sia diverso da 0 (il che indica un fallimento).

``\&\&'' invece esegue il comando se il precedente esce con un \emph{exit
code} uguale a 0 (cio\`e esce con successo).

Esempi:
\begin{verbatim}
$ gcc prova.c && ./a.out
$ gcc prova.c || echo Compilazione fallita
\end{verbatim}

\section{Raggruppamento di comandi}
\label{sec:sub:raggr}
\`E possibile raggruppare dei comandi all'interno di ``('' e ``)'': verranno
eseguiti in una subshell, e condivideranno gli stessi \texttt{stdin},
\texttt{stdout} e \texttt{stderr}:
\begin{verbatim}
$ date ; ls ; pwd > out.txt
$ (date ; ls ; pwd) > out.txt
\end{verbatim}

\section{Esecuzione in background}
\label{sec:sub:backg}
Se il comando lanciato da shell viene seguito dal metacarattere ``\&'', viene
creata una subshell, il comando viene avviato in background e il controllo
torna alla shell immediatamente, che prosegue l'esecuzione concorrentemente
con il processo lanciato. Quest'ultimo abbandoner\`a la sorgente di
\texttt{stdin} come tastiera (quindi non sar\`a possibile fornirgli input in
tale modo), ritornadolo alla shell che vi permetter\`a di proseguire il
lavoro; ovviamente \`e una funzionalit\`a molto utile se dovete eseguire
attivit\`a lunghe che non necessitano il vostro input:
\begin{verbatim}
$ find / -name passwd -print &
 [34256]
$ /etc/passwd
$ find / -name passwd -print &> results.txt &
 [34263]
$
\end{verbatim}

\section{Quoting}
Esiste la possibilit\`a di disabilitare \emph{command substitution},
\emph{wildcards} e \emph{variable substitution} (per non essere costretti a
fare necessariamente l'escape di ogni carattere) attraverso il \emph{quoting}.

Con gli apici singoli (\texttt{'{}'}) vengono inibiti \emph{wildcards},
\emph{command substitution} e \emph{variable substitution}:
\begin{verbatim}
$ echo 3 * 4 = 12
$ echo '3 * 4 = 12'
\end{verbatim}

Con i doppi apici (\texttt{""}), invece, vengono disabilitate le sole
\emph{wildcards}:
\begin{verbatim}
$ export nome="mionome"
$ echo "Il mio nome: $nome - la data: `date`"
$ echo 'Il mio nome: $nome - la data: `date`'
\end{verbatim}

\section{Subshell}
Quando aprite un terminale (ad esempio quando fate login) viene eseguita una
\emph{shell}. Viene creata una \emph{child shell} (o \emph{subshell}) quando
\begin{itemize}
	\item usate il raggruppamento di comandi (vedi \ref{sec:sub:raggr})
	\item eseguite un processo in background (vedi \ref{sec:sub:backg})
	\item eseguite uno script
\end{itemize}
Le \emph{subshell} hanno la propria directory corrente, la propria area di
variabili indipendenti dalla \emph{shell} madre (vedi \ref{sec:sub:var} per le
variabili).

\section{Variabili}
\label{sec:sub:var}
Le variabili supportate da qualsiasi shell sono di due tipi: \emph{globali} e
\emph{locali}.

Le variabili \emph{locali} non vengono passate da una shell alle altre
subshell.

Al contrario, le variabili \emph{globali} (chiamate anche variabili
\emph{d'ambiente}) vengono passate da una shell alle subshell che essa crea, e
vengono utilizzate per comunicazioni tra shell madre e figlie.

Entrambi i tipi di variabile contengono dati di tipo \textit{stringa}. Ogni
shell ha una serie di variabili d'ambiente che vengono inizializzate all'avvio
della shell stessa (in genere attraverso i file \textit{/etc/profile},
\textit{/home/nomeutente/.bash\_profile}, etc.), come:
\begin{verbatim}
$HOME, $PATH, $USER, $SHELL, $TERM, ...
\end{verbatim}
Un elenco completo delle variabili attualmente in uso si ottiene attraverso il
comando \texttt{env}.

Per accedere al contenuto di una variabile, si pu\`o utilizzare il comando \$;
\$VARIABILE \`e la forma abbreviata di \$\{VARIABILE\} (a volte quest'ultima
forma \`e necessaria):
\begin{verbatim}
$ echo $SHELL
 /bin/bash
\end{verbatim}

Per assegnare un valore ad una variabile, nel caso si tratti di una variabile
locale, basta il nome della variabile seguito dal valore\footnote{fate
attenzione, perch\'e variabile e valore devono essere inframezzate da un segno
di uguale (=), ma tra il segno e i caratteri \textbf{non ci devono essere
spazi}}:
\begin{verbatim}
$ mionome=Ciro\ Mattia\ Gonano   # problemi con gli spazi
$ mionome="Ciro Mattia Gonano"   # nessun problema ;)
\end{verbatim}
Per trasformare una variabile locale dichiarata in questo modo, utilizzare il
comando \texttt{export}\footnote{\texttt{export} funziona solo se usate una
shell di tipo \emph{bourne} (come \emph{bash}), mentre se utilizzate una
\emph{c shell} (come \emph{csh}) dovete usare il comando \texttt{setenv}}:
\begin{verbatim}
$ mionome="Ciro Mattia Gonano"
$ export mionome
\end{verbatim}

Un esempio pi\`u lungo per capire come funzionano le variabili locali e
d'ambiente:
\begin{verbatim}
$ nome="Ciro Mattia"
$ cognome="Gonano"
$ echo $nome $cognome
$ export nome
$ bash
$ echo $nome $cognome
$ exit
$ echo $nome $cognome
\end{verbatim}

\section{Scripting}
Uno script \`e una sequenza di comandi che devono essere eseguiti (e,
ovviamente, pu\`o contenere anche sequenze condizionali), memorizzata
all'interno di un file di testo e poi richiamato dalla shell.
Risulta molto utile per automatizzare procedure lunghe che ripetete
frequentemente.

Per scrivere uno script, \`e sufficiente scrivere la sequenza di comandi nel
file di testo, renderlo eseguibile (con \texttt{chmod}), e
lanciarlo\footnote{state attenti al PATH: se la directory corrente non \`e
all'interno della vostra variabile \texttt{\$PATH}, dovrete lanciare qualcosa
di simile a \texttt{./mioscript}}; la shell si occuper\`a di decidere la shell
con cui eseguire lo script, di creare la subshell, e di passare il contenuto
del file come \texttt{stdin} della subshell.

La selezione della shell da usare per lo script \`e basata sulla prima riga
dello stesso script:
\begin{itemize}
	\item se contiene solo un simbolo \texttt{\#}, verr\`a utilizzata la
		shell corrente;
	\item se contiene \texttt{\#!\textit{pathname}}, verr\`a utilizzata la
		shell identificata da \textit{pathname}: per eseguire uno
		script con la Korn shell scriveremo nella prima riga dello
		script \texttt{\#!/bin/ksh}. Questa \`e la forma che vi
		consiglio di utilizzare sempre, perch\'e risulta non ambigua e
		non dipende dalla configurazione su cui lo script viene
		lanciato.
	\item se non contiene nessuno di questi due, viene interpretato da una
		Bourne shell (\emph{bash}, \emph{sh}, etc.).
\end{itemize}
Alcuni comportamenti interessanti per farvi capire come viene eseguito uno
script\footnote{eseguiteli senza paura, non causeranno danni al vostro
sistema, se li eseguite da normale utente}...
\begin{verbatim}
#!/bin/cat
#!/bin/rm
\end{verbatim}

Esistono alcune variabili \emph{built-in} studiate apposta per gli script (di
seguito sono indicate se valgono per le Bourne shell (\textit{sh}), per le C
shell (\textit{csh}) o per entrambe):
\begin{itemize}
	\item \texttt{\$\$\ \ \ \ \ \ } l'identificatore di processo della
		shell (\textit{sh})
	\item \texttt{\$0\ \ \ \ \ \ } il nome dello shell script
		(\textit{sh,csh})
	\item \texttt{\$1-\$9\ \ \ } l'n-esimo argomento della riga di comando
		(\textit{sh,csh})
	\item \texttt{\$\{n\}\ \ \ \ } l'n-esimo argomento della riga di
		comando (\textit{sh,csh})
	\item \texttt{\$*\ \ \ \ \ \ } lista di tutti gli argomenti della riga
		di comando (\textit{sh,csh})
	\item \texttt{\$\#\ \ \ \ \ \ } numero degli argomenti della riga di
		comando (\textit{sh})
	\item \texttt{\$?\ \ \ \ \ \ } valore di uscita dell'ultimo comando
		eseguito (\textit{sh})
\end{itemize}

Un piccolo script per cominciare a capire...
\begin{verbatim}
#!/bin/bash
a=23
echo $a
b=$a
echo $b
# Questo e` un commento!
# Proviamo qualcosa di piu` eccitante!
a=`echo Ciao mondo\!` # assegna il risultato di echo ad a
echo $a
a=`ls -l`  # adesso a e` uguale all'output
           # di ls -l
echo $a
exit 0
\end{verbatim}

\section{Here document}
L'ultimo metacarattere che ci rimane da vedere \`e il cosiddetto ``Here
document'' (``\texttt{<<}''). I comandi
\begin{verbatim}
<comando> << <parola>

<comando> <</ <parola>
\end{verbatim}
servono a copiare il proprio \texttt{stdin} fino alla linea che inizia con
\texttt{<parola>} (esclusa), e quindi eseguire \texttt{<command>} utilizzando
questi dati copiati come \texttt{stdin}.

La versione con la \textit{slash} (``\texttt{<</}'') non esegue variable
substitution.

Esempio:
\begin{verbatim}
$ cat << ENDOFTEXT > nota
> Ricordarsi di specificare che
> con << le variabili vengono sostituite
> e con <</ no.
>                  $mionome
> ENDOFTEXT
$ 
\end{verbatim}

