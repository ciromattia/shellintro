\chapter{Introduzione}

\section{Cos'\`e la shell di Linux}
La shell di Linux costituisce l'interfaccia tra il sistema e l'utente umano.

Consente di:
\begin{itemize}
	\item poter interagire con il sistema a svariati livelli
	\item richiamare altri programmi
	\item programmare degli script\footnote{gli script sono programmi in
		formato testuale formati da sequenze di comandi, fondamentali
		per ottimizzare e semplificare la vita all'amministratore del
		sistema}
\end{itemize}

In ambiente Linux abbiamo a disposizione svariati tipi di shells: \emph{sh, ash,
csh, bash, esh, lsh, kiss, pdksh, sash, psh, tcsh, zsh} tanto per citarne solo
alcune. Alcune sono derivate (e potenziate) da altre\footnote{come \emph{bash} \`e
una versione potenziata di \emph{sh}}, mentre altre differiscono in maniera
notevole, sia nell'uso che nell'impostazione verso l'utente.

Per la grande variet\`a presente, qui ci concentreremo sull'utilizzo di
\emph{bash}, che \`e ormai lo standard "de facto" sui sistemi
Linux\footnote{attenzione: NON Unix/BSD, per i quali lo standard \`e in genere
\emph{csh} o una sua derivata}.

\section{Prerequisiti}
Lo shell scripting non \`e materia astrusa, ma nemmeno banale. Nel corso di
queste pagine daremo per scontate alcune nozioni che riteniamo fondamentali
per un utente Linux, su cui non ci soffermeremo per non appesantire il
documento.
In particolare daremo per scontata la conoscenza di:
\begin{itemize}
	\item nozioni di login/logout;
	\item organizzazione del filesystem (file e directory, gerarchie,
	pathname assoluti e relativi);
	\item comandi di base:
	\begin{itemize}
		\item listare il contenuto di una directory (\texttt{ls});
		\item maneggiare i file (\texttt{cp}, \texttt{mv}, \texttt{rm}, 
			\texttt{mkdir}, \texttt{rmdir});
		\item visualizzare il contenuto di un file
		(\texttt{cat/more/less/head/tail});
	\end{itemize}
	\item attributi dei file (\texttt{read/write/execute})
	\item conoscenza del significato di permesso (\texttt{chmod})
	\item gestione dei gruppi e dei proprietari dei file (\texttt{chgrp/chown})
\end{itemize}
per cui, se non vi sentite sicuri di questi concetti, vi consigliamo di
rivederli prima di avvicinarvi al bash scripting.

\section{Note tipografiche}
Per facilitare la lettura e la comprensione del testo, sono state utilizzate
le seguenti convenzioni tipografiche:
\begin{itemize}
	\item lo stile {\ttfamily macchina da scrivere} \`e utilizzato per i
		comandi, le variabili, le parole chiave e gli script;
	\item in {\itshape corsivo} le definizioni di nuovi termini,
		gli argomenti dei comandi e i nomi dei file;
%	\item lo stile {\sffamily sans serif}
\end{itemize}

