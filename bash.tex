\documentclass[a4paper,10pt]{report}
\pagestyle{headings}

\title{\textsc{\textbf{Introduzione alla shell}}}

\author{Ciro Mattia Gonano <ciro@winged.it>}

\date{}

\usepackage[italian]{babel}
\usepackage[T1]{fontenc}
\frenchspacing
%\usepackage{longtable}

\begin{document}

\maketitle

\section{Disclaimer}

Copyright (c) 2003-2006 Ciro Mattia Gonano.
Permission is granted to copy, distribute and/or modify this document under
the terms of the GNU Free Documentation License, Version 1.2 or any later
version published by the Free Software Foundation; with no Invariant Sections,
no Front-Cover Texts, and no Back-Cover Texts.

A copy of the license is included in the section entitled "GNU Free
Documentation License".

\medskip
Vi sarei grato se riportaste qualsiasi errore, imprecisione, difficolt\`a
trovaste nel testo. Contribuirete all'evoluzione della presente guida, e molta
altra gente ve ne sar\`a grata ;-)

\section{Ringraziamenti}
La prima stesura del presente documento \`e avvenuta prendendo spunto per la
sua quasi totale integrit\`a dai lucidi del corso di ``Laboratorio di Sistemi
Operativi'' del professor \textbf{Alberto Montresor}.

\section{Nota alla presente versione}

Questa versione (generata in data \date) non ha subito cambiamenti da due anni.

Questo implica che, come in ogni cosa che riguardi l'informatica, diverse cose
potrebbero essere cambiate o migliorate; inoltre la presente guida \`e tutto
fuorch\'e finita.

Ho deciso di pubblicarla unicamente per dare un ``la'' a chi fosse interessato
ad approfondire l'utilizzo della shell, ed eventualmente come spunto per chi
volesse contribuire ad ampliare e completare la guida stessa.

Ovviamente i sorgenti sono disponibili su richiesta.

\tableofcontents

% Chapter 1: Introduzione e note tipografiche
\chapter{Introduzione}

\section{Cos'\`e la shell di Linux}
La shell di Linux costituisce l'interfaccia tra il sistema e l'utente umano.

Consente di:
\begin{itemize}
	\item poter interagire con il sistema a svariati livelli
	\item richiamare altri programmi
	\item programmare degli script\footnote{gli script sono programmi in
		formato testuale formati da sequenze di comandi, fondamentali
		per ottimizzare e semplificare la vita all'amministratore del
		sistema}
\end{itemize}

In ambiente Linux abbiamo a disposizione svariati tipi di shells: \emph{sh, ash,
csh, bash, esh, lsh, kiss, pdksh, sash, psh, tcsh, zsh} tanto per citarne solo
alcune. Alcune sono derivate (e potenziate) da altre\footnote{come \emph{bash} \`e
una versione potenziata di \emph{sh}}, mentre altre differiscono in maniera
notevole, sia nell'uso che nell'impostazione verso l'utente.

Per la grande variet\`a presente, qui ci concentreremo sull'utilizzo di
\emph{bash}, che \`e ormai lo standard "de facto" sui sistemi
Linux\footnote{attenzione: NON Unix/BSD, per i quali lo standard \`e in genere
\emph{csh} o una sua derivata}.

\section{Prerequisiti}
Lo shell scripting non \`e materia astrusa, ma nemmeno banale. Nel corso di
queste pagine daremo per scontate alcune nozioni che riteniamo fondamentali
per un utente Linux, su cui non ci soffermeremo per non appesantire il
documento.
In particolare daremo per scontata la conoscenza di:
\begin{itemize}
	\item nozioni di login/logout;
	\item organizzazione del filesystem (file e directory, gerarchie,
	pathname assoluti e relativi);
	\item comandi di base:
	\begin{itemize}
		\item listare il contenuto di una directory (\texttt{ls});
		\item maneggiare i file (\texttt{cp}, \texttt{mv}, \texttt{rm}, 
			\texttt{mkdir}, \texttt{rmdir});
		\item visualizzare il contenuto di un file
		(\texttt{cat/more/less/head/tail});
	\end{itemize}
	\item attributi dei file (\texttt{read/write/execute})
	\item conoscenza del significato di permesso (\texttt{chmod})
	\item gestione dei gruppi e dei proprietari dei file (\texttt{chgrp/chown})
\end{itemize}
per cui, se non vi sentite sicuri di questi concetti, vi consigliamo di
rivederli prima di avvicinarvi al bash scripting.

\section{Note tipografiche}
Per facilitare la lettura e la comprensione del testo, sono state utilizzate
le seguenti convenzioni tipografiche:
\begin{itemize}
	\item lo stile {\ttfamily macchina da scrivere} \`e utilizzato per i
		comandi, le variabili, le parole chiave e gli script;
	\item in {\itshape corsivo} le definizioni di nuovi termini,
		gli argomenti dei comandi e i nomi dei file;
%	\item lo stile {\sffamily sans serif}
\end{itemize}



% Chapter 2: Nozioni di base
\chapter{Nozioni di base}

\section{Caratteristiche della shell}
Ogni shell (non solo bash) presenta una serie di caratteristiche, raggruppabili in macrocategorie:
\begin{itemize}
	\item comandi built-in;
	\item metacaratteri;
	\item redirezione di I/O;
	\item command substitution;
	\item sequenze condizionali e non;
	\item raggruppamento di comandi;
	\item esecuzione in background;
	\item quoting;
	\item subshell;
	\item variabili (locali/d'ambiente);
	\item scripting.
\end{itemize}
Passiamo a vederli in generale.

\section{Comandi built-in}
Quando richiedete l'esecuzione di un comando esterno, la shell si
occupa di richiamare il corrispondente file eseguibile, caricarlo in memoria
ed eseguirlo (ad esempio, quando chiamiamo il comando \texttt{ls},
\textit{bash} si occupa di trovare e lanciare \texttt{/bin/ls}).

Al contrario, quando richiedete un comando che possa essere riconosciuto ed
eseguito direttamente dalla shell, fate uso di un comando \textit{interno},
detto anche \textit{built-in}.\\
Esempi:
\begin{verbatim}
  echo         # stampa tutti i suoi argomenti sullo standard output
  cd           # cambia directory
\end{verbatim}

\section{Metacaratteri}
La shell dispone di una serie di caratteri che non sono considerati nella loro
forma a video, ma servono a specificare serie di funzionalit\`a interne alla
shell. Per questo motivo quando volete digitare uno di tali caratteri dovete
solitamente farne l'\textit{escape} con il carattere \textit{backslash}
(``$\backslash$'').

I metacaratteri possono essere trattati in maniera leggermente diversa a
seconda della shell utilizzata, ma in generale sono questi\footnote{e
\textit{bash} segue questo elenco}:
\begin{itemize}
	\item \texttt{>, >>, <\ \ \ \ \ \ \ \ } redirezione I/O
	\item \texttt{|\ \ \ \ \ \ \ \ \ \ \ \ \ \ } pipe
	\item \texttt{*, ?, [\ldots]\ \ \ \ } wildcards
	\item \texttt{`command`\ \ \ \ \ \ } command substitution
	\item \texttt{;\ \ \ \ \ \ \ \ \ \ \ \ \ \ } esecuzione sequenziale
	\item \texttt{||, \&\&\ \ \ \ \ \ \ \ \ } esecuzione condizionale
	\item \texttt{(\ldots)\ \ \ \ \ \ \ \ \ \ } raggruppamento comandi
	\item \texttt{\&\ \ \ \ \ \ \ \ \ \ \ \ \ \ } esecuzione in background
	\item \texttt{"", '{}'\ \ \ \ \ \ \ \ \ \ } quoting
	\item \texttt{\#\ \ \ \ \ \ \ \ \ \ \ \ \ \ } commento
	\item \texttt{\$\ \ \ \ \ \ \ \ \ \ \ \ \ \ } espansione di variabile
	\item \texttt{$\backslash$\ \ \ \ \ \ \ \ \ \ \ \ \ \ } carattere di backslash
	\item \texttt{<<\ \ \ \ \ \ \ \ \ \ \ \ \ \ } ``here documents''
\end{itemize}

\section{Redirezione I/O}
Ogni processo \`e associato a tre ``stream'': lo \textit{standard input}
(\texttt{stdin}), lo \textit{standard output} (\texttt{stdout}) e lo
\textit{standard error} (\texttt{stderr}).

La redirezione dell'I/O permette di associare questi stream a
sorgenti/destinazioni\footnote{ovviamente nel caso di \texttt{stdin} avremo
una sorgente, mentre per \texttt{stdout} e \texttt{stderr} avremo due
destinazioni -- che \emph{possono} coincidere} diverse dalle loro attuali.
Possiamo in questo modo salvare l'output di un processo in un file di testo,
oppure possiamo salvare il suo log di errore, attraverso semplici comandi si
questo tipo:
\begin{verbatim}
  ls > list.txt     # salva l'output di ls in un file list.txt
  ls >> list.txt    # aggiunge l'output di ls in coda
                    # al file list.txt
  rm fileinutile >& /dev/null  # redireziona stdout e stderr
                               # di rm su /dev/null
\end{verbatim}
Notate che, se non specificato, ``>'' ridirige lo \texttt{stdout}, e ``<'' lo
\texttt{stdin}. ``>\&'' ridirige \emph{tutti} i flussi di output (quindi anche
lo \texttt{stderr}). Notate anche il diverso significato di ``>'' e ``>>'': il
primo se rediretto in un file lo tronca al suo inizio, cancellando il suo
contenuto prima di scrivere il flusso, mentre il secondo conserva il contenuto
del file e accoda ad esso il flusso.

\addcontentsline{toc}{subsection}{Pipe, o catene di montaggio}
\subsection*{Pipe, o catene di montaggio}
Attraverso le pipe, potete utilizzare l'output di un processo come input di un
altro processo:
\begin{verbatim}
$ ls
  a.c b.c a.o b.o dir1 dir2
$ ls | wc -w
  6
\end{verbatim}
\texttt{wc} (un programma che conta parole, linee e molto altro) prende come
\texttt{stdin} lo \texttt{stdout} di \texttt{ls}, ritornando il numero di
parole listate da quest'ultimo: 6, appunto.

\medskip
Potrebbe sembrare una grossa limitazione poter unicamente ridirigere il
flusso, e non poterlo duplicare: a questo supplisce il comando \texttt{tee}
\footnote{il nome deriva dai ``giunti a T'', utilizzati dagli idraulici}
che copia lo \texttt{stdin} sul file specificato E sullo \texttt{stdout}:
\begin{verbatim}
$ who | tee who_capture.txt | sort
\end{verbatim}
Se \texttt{tee} viene chiamato con l'opzione \texttt{-a}, viene fatto l'append
al file argomento, invece del troncamento.

\section{Wildcards}
Le wildcards sono utilizzate per specificare pattern comprendenti pi\`u file:
i file vengono passati in rassegna, e quelli che corrispondono\footnote{in
gergo, ``\emph{to match}''} vengono restituiti. Vengono utlizzati pi\`u spesso
delle \textit{regular expressions} per la loro semplicit\`a, ad ogni modo sono
un sottoinsieme di queste ultime.
\begin{itemize}
	\item \texttt{*\ \ \ \ \ } match di qualsiasi stringa (anche vuota)
	\item \texttt{?\ \ \ \ \ } match di qualsiasi carattere singolo
	\item \texttt{[\ldots]\ } match di qualsiasi carattere inserito nelle
		parentesi
\end{itemize}
Esempi:
\begin{verbatim}
$ ls *.c
 prova001.c prova004.c prova125.c prov.c
$ ls prova00?.c
 prova001.c prova004.c
$ ls prova[0-9][0-9][0-9].c
 prova001.c prova004.c prova125.c
\end{verbatim}

\section{Command substitution}
Si pu\`o sostituire ad un argomento un comando del tipo
\texttt{`comando`}: il suo standard output verr\`a sostituito al comando
stesso:
\begin{verbatim}
$ echo Data odierna: `date`
$ echo Utenti collegati: `who | wc -l`
$ tar czvf src-`date`.tar.gz src/
\end{verbatim}

\section{Sequenze}
\addcontentsline{toc}{subsection}{Non condizionali}
\subsection*{Non condizionali}
Si possono eseguire sequenzialmente dei comandi utilizzando il metacarattere
``;'':
\begin{verbatim}
$ date ; pwd ; ls
\end{verbatim}

\addcontentsline{toc}{subsection}{Condizionali}
\subsection*{Condizionali}
Le sequenze condizionali specificano la sequenza da eseguire in base
all'\emph{exit code} del comando precedente all'interno della sequenza.

``||'' viene utilizzato per eseguire il comando nel caso l'\emph{exit code}
del precedente sia diverso da 0 (il che indica un fallimento).

``\&\&'' invece esegue il comando se il precedente esce con un \emph{exit
code} uguale a 0 (cio\`e esce con successo).

Esempi:
\begin{verbatim}
$ gcc prova.c && ./a.out
$ gcc prova.c || echo Compilazione fallita
\end{verbatim}

\section{Raggruppamento di comandi}
\label{sec:sub:raggr}
\`E possibile raggruppare dei comandi all'interno di ``('' e ``)'': verranno
eseguiti in una subshell, e condivideranno gli stessi \texttt{stdin},
\texttt{stdout} e \texttt{stderr}:
\begin{verbatim}
$ date ; ls ; pwd > out.txt
$ (date ; ls ; pwd) > out.txt
\end{verbatim}

\section{Esecuzione in background}
\label{sec:sub:backg}
Se il comando lanciato da shell viene seguito dal metacarattere ``\&'', viene
creata una subshell, il comando viene avviato in background e il controllo
torna alla shell immediatamente, che prosegue l'esecuzione concorrentemente
con il processo lanciato. Quest'ultimo abbandoner\`a la sorgente di
\texttt{stdin} come tastiera (quindi non sar\`a possibile fornirgli input in
tale modo), ritornadolo alla shell che vi permetter\`a di proseguire il
lavoro; ovviamente \`e una funzionalit\`a molto utile se dovete eseguire
attivit\`a lunghe che non necessitano il vostro input:
\begin{verbatim}
$ find / -name passwd -print &
 [34256]
$ /etc/passwd
$ find / -name passwd -print &> results.txt &
 [34263]
$
\end{verbatim}

\section{Quoting}
Esiste la possibilit\`a di disabilitare \emph{command substitution},
\emph{wildcards} e \emph{variable substitution} (per non essere costretti a
fare necessariamente l'escape di ogni carattere) attraverso il \emph{quoting}.

Con gli apici singoli (\texttt{'{}'}) vengono inibiti \emph{wildcards},
\emph{command substitution} e \emph{variable substitution}:
\begin{verbatim}
$ echo 3 * 4 = 12
$ echo '3 * 4 = 12'
\end{verbatim}

Con i doppi apici (\texttt{""}), invece, vengono disabilitate le sole
\emph{wildcards}:
\begin{verbatim}
$ export nome="mionome"
$ echo "Il mio nome: $nome - la data: `date`"
$ echo 'Il mio nome: $nome - la data: `date`'
\end{verbatim}

\section{Subshell}
Quando aprite un terminale (ad esempio quando fate login) viene eseguita una
\emph{shell}. Viene creata una \emph{child shell} (o \emph{subshell}) quando
\begin{itemize}
	\item usate il raggruppamento di comandi (vedi \ref{sec:sub:raggr})
	\item eseguite un processo in background (vedi \ref{sec:sub:backg})
	\item eseguite uno script
\end{itemize}
Le \emph{subshell} hanno la propria directory corrente, la propria area di
variabili indipendenti dalla \emph{shell} madre (vedi \ref{sec:sub:var} per le
variabili).

\section{Variabili}
\label{sec:sub:var}
Le variabili supportate da qualsiasi shell sono di due tipi: \emph{globali} e
\emph{locali}.

Le variabili \emph{locali} non vengono passate da una shell alle altre
subshell.

Al contrario, le variabili \emph{globali} (chiamate anche variabili
\emph{d'ambiente}) vengono passate da una shell alle subshell che essa crea, e
vengono utilizzate per comunicazioni tra shell madre e figlie.

Entrambi i tipi di variabile contengono dati di tipo \textit{stringa}. Ogni
shell ha una serie di variabili d'ambiente che vengono inizializzate all'avvio
della shell stessa (in genere attraverso i file \textit{/etc/profile},
\textit{/home/nomeutente/.bash\_profile}, etc.), come:
\begin{verbatim}
$HOME, $PATH, $USER, $SHELL, $TERM, ...
\end{verbatim}
Un elenco completo delle variabili attualmente in uso si ottiene attraverso il
comando \texttt{env}.

Per accedere al contenuto di una variabile, si pu\`o utilizzare il comando \$;
\$VARIABILE \`e la forma abbreviata di \$\{VARIABILE\} (a volte quest'ultima
forma \`e necessaria):
\begin{verbatim}
$ echo $SHELL
 /bin/bash
\end{verbatim}

Per assegnare un valore ad una variabile, nel caso si tratti di una variabile
locale, basta il nome della variabile seguito dal valore\footnote{fate
attenzione, perch\'e variabile e valore devono essere inframezzate da un segno
di uguale (=), ma tra il segno e i caratteri \textbf{non ci devono essere
spazi}}:
\begin{verbatim}
$ mionome=Ciro\ Mattia\ Gonano   # problemi con gli spazi
$ mionome="Ciro Mattia Gonano"   # nessun problema ;)
\end{verbatim}
Per trasformare una variabile locale dichiarata in questo modo, utilizzare il
comando \texttt{export}\footnote{\texttt{export} funziona solo se usate una
shell di tipo \emph{bourne} (come \emph{bash}), mentre se utilizzate una
\emph{c shell} (come \emph{csh}) dovete usare il comando \texttt{setenv}}:
\begin{verbatim}
$ mionome="Ciro Mattia Gonano"
$ export mionome
\end{verbatim}

Un esempio pi\`u lungo per capire come funzionano le variabili locali e
d'ambiente:
\begin{verbatim}
$ nome="Ciro Mattia"
$ cognome="Gonano"
$ echo $nome $cognome
$ export nome
$ bash
$ echo $nome $cognome
$ exit
$ echo $nome $cognome
\end{verbatim}

\section{Scripting}
Uno script \`e una sequenza di comandi che devono essere eseguiti (e,
ovviamente, pu\`o contenere anche sequenze condizionali), memorizzata
all'interno di un file di testo e poi richiamato dalla shell.
Risulta molto utile per automatizzare procedure lunghe che ripetete
frequentemente.

Per scrivere uno script, \`e sufficiente scrivere la sequenza di comandi nel
file di testo, renderlo eseguibile (con \texttt{chmod}), e
lanciarlo\footnote{state attenti al PATH: se la directory corrente non \`e
all'interno della vostra variabile \texttt{\$PATH}, dovrete lanciare qualcosa
di simile a \texttt{./mioscript}}; la shell si occuper\`a di decidere la shell
con cui eseguire lo script, di creare la subshell, e di passare il contenuto
del file come \texttt{stdin} della subshell.

La selezione della shell da usare per lo script \`e basata sulla prima riga
dello stesso script:
\begin{itemize}
	\item se contiene solo un simbolo \texttt{\#}, verr\`a utilizzata la
		shell corrente;
	\item se contiene \texttt{\#!\textit{pathname}}, verr\`a utilizzata la
		shell identificata da \textit{pathname}: per eseguire uno
		script con la Korn shell scriveremo nella prima riga dello
		script \texttt{\#!/bin/ksh}. Questa \`e la forma che vi
		consiglio di utilizzare sempre, perch\'e risulta non ambigua e
		non dipende dalla configurazione su cui lo script viene
		lanciato.
	\item se non contiene nessuno di questi due, viene interpretato da una
		Bourne shell (\emph{bash}, \emph{sh}, etc.).
\end{itemize}
Alcuni comportamenti interessanti per farvi capire come viene eseguito uno
script\footnote{eseguiteli senza paura, non causeranno danni al vostro
sistema, se li eseguite da normale utente}...
\begin{verbatim}
#!/bin/cat
#!/bin/rm
\end{verbatim}

Esistono alcune variabili \emph{built-in} studiate apposta per gli script (di
seguito sono indicate se valgono per le Bourne shell (\textit{sh}), per le C
shell (\textit{csh}) o per entrambe):
\begin{itemize}
	\item \texttt{\$\$\ \ \ \ \ \ } l'identificatore di processo della
		shell (\textit{sh})
	\item \texttt{\$0\ \ \ \ \ \ } il nome dello shell script
		(\textit{sh,csh})
	\item \texttt{\$1-\$9\ \ \ } l'n-esimo argomento della riga di comando
		(\textit{sh,csh})
	\item \texttt{\$\{n\}\ \ \ \ } l'n-esimo argomento della riga di
		comando (\textit{sh,csh})
	\item \texttt{\$*\ \ \ \ \ \ } lista di tutti gli argomenti della riga
		di comando (\textit{sh,csh})
	\item \texttt{\$\#\ \ \ \ \ \ } numero degli argomenti della riga di
		comando (\textit{sh})
	\item \texttt{\$?\ \ \ \ \ \ } valore di uscita dell'ultimo comando
		eseguito (\textit{sh})
\end{itemize}

Un piccolo script per cominciare a capire...
\begin{verbatim}
#!/bin/bash
a=23
echo $a
b=$a
echo $b
# Questo e` un commento!
# Proviamo qualcosa di piu` eccitante!
a=`echo Ciao mondo\!` # assegna il risultato di echo ad a
echo $a
a=`ls -l`  # adesso a e` uguale all'output
           # di ls -l
echo $a
exit 0
\end{verbatim}

\section{Here document}
L'ultimo metacarattere che ci rimane da vedere \`e il cosiddetto ``Here
document'' (``\texttt{<<}''). I comandi
\begin{verbatim}
<comando> << <parola>

<comando> <</ <parola>
\end{verbatim}
servono a copiare il proprio \texttt{stdin} fino alla linea che inizia con
\texttt{<parola>} (esclusa), e quindi eseguire \texttt{<command>} utilizzando
questi dati copiati come \texttt{stdin}.

La versione con la \textit{slash} (``\texttt{<</}'') non esegue variable
substitution.

Esempio:
\begin{verbatim}
$ cat << ENDOFTEXT > nota
> Ricordarsi di specificare che
> con << le variabili vengono sostituite
> e con <</ no.
>                  $mionome
> ENDOFTEXT
$ 
\end{verbatim}



% Chapter 3: Programmazione della shell
\chapter{Programmazione della shell}

\section{Valutazione delle espressioni}
La shell non supporta direttamente (come faceva, per chi lo ricorda, il caro vecchio
\textbf{C$=$64}, avendo il \emph{BASIC} integrato) la valutazione delle
espressioni; a questo supplisce il comando \texttt{expr \textit{espressione}},
che ci ritorna il risultato della formula introdotta. Tutti i componenti
dell'espressione devono essere necessariamente separati da spazi tra di loro,
e tutti i metacaratteri devono essere ovviamente "escaped"\footnote{come
dicevamo prima, apponendo una \textit{backslash} ($\backslash$) al carattere
(ad esempio: $\backslash$*)}, altrimenti la shell li espander\`a secondo il
loro metasignificato.

Il risultato pu\`o essere numerico o una stringa, e pu\`o venire assegnato ad
una variabile con un uso opportuno del \emph{command substitution}.

Gli operatori disponibili sono:
\begin{itemize}
	\item Aritmetici:\texttt{\ \ \ \ + - * / \%}
	\item Confronto:\texttt{\ \ \ \ = != > < >= <=}
	\item Logici:\texttt{\ \ \ \ \ \ \ \ \& | !}
	\item Parentesi\footnote{devono essere prefissate dalla
	\emph{backslash}}: \texttt{\ \ \ ( )}
	\item Stringhe:
	\begin{itemize}
		\item \texttt{match \textit{string regexp}}
		\item \texttt{substr \textit{string start length}}
		\item \texttt{length \textit{string}}
		\item \texttt{index \textit{string charList}}
	\end{itemize}
\end{itemize}

\section{Exit status}
Ogni comando lanciato dalla shell ritorna un \emph{exit status}. Per
convenzione UNIX, un \emph{exit status} pari a 0 indica ``uscita con
successo'' (pari al valore \texttt{TRUE} nella valutazione delle espressioni
booleane), mentre un valore diverso indica ``uscita (spesso prematura) con
fallimento''. L'\emph{exit status} torna molto utile negli script per
controllare il flusso dell'esecuzione dei comandi, modificandolo in base ai
risutlati d'uscita prodotti. Proprio per questo, la shell ci offre il comando
\texttt{exit \textit{nn}} per far terminare il nostro script con \emph{exit
status} pari a \textit{nn}, e la variabile \emph{built-in} \texttt{\$?}, che
contiene l'\emph{exit status} dell'ultimo comando eseguito:
\begin{verbatim}
$ cat script.sh
#!/bin/bash
echo hello
echo $?     # "echo hello" ha tornato exit status 0 (successo)
lskdf       # comando inesistente
echo $?     # "lskdf" torna exit status diverso da 0 (fallimento)
exit 113    # il nostro script esce tornando exit status
            # 113 (attenzione: viene considerato fallimento!)
$ chmod 755 script.sh
$ ./script.sh
hello
0
./script.sh: lskdf: command not found
127
\end{verbatim}

\section{Strutture di controllo}
Come dicevamo prima, gli \emph{exit status} vengono utilizzati per le
espressioni condizionali che governano il programma:
\begin{verbatim}
if cmp a b > /dev/null   # ridirigiamo l'output su /dev/null
then echo "I file a e b sono identici."
else echo "I file a e b sono diversi."
fi
\end{verbatim}

Le condizioni sono prevalentemente controllate con il comando \texttt{test
\textit{expression}}, che torna 0 se l'espressione \`e vera, 1 altrimenti.
Molti sono gli argomenti di confronto che pu\`o accettare \texttt{test}, e
sarebbe inutile elencarli qui, quindi vi rimando alla manpage \textbf{test(1)}.
\`E utile comunque ricordare che gli operatori di confronto sono diversi per
interi e stringhe, e che esistono operatori di test per file e directory (ad
esempio, per sapere se un file esiste ed e` eseguibile).

La shell offre il comando built-in \verb_[ ]_ e le keyword
\verb_[[ ]]_ per valutare la condizione (usando il valore di
ritorno di \texttt{test}) al loro interno. La differenza tra i
due sta nel fatto che all'interno di \verb_[[ ]]_ non viene
effettuato filename expansion, inoltre operatori come \&\&, ||,
> e < vengono interpretati correttamente.

Ad esempio, per controllare l'esistenza di un argomento:
\begin{verbatim}
if [ -n "$1" ]
then
  lines=$1
fi
\end{verbatim}

Questo esempio ci introduce il cosiddetto \textit{blocco condizionale},
comunemente chiamato \textit{blocco if}.
Un blocco condizionale permette di eseguire comandi condizionati, ovvero di
scegliere, a seconda che una condizione sia vera o falsa, una lista di comandi
piuttosto che un'altra. La struttura di un blocco condizionale \`e la
seguente:
\begin{verbatim}
if list1
then
         list2
elif list3
then
         list4
else
         list5
fi
\end{verbatim}

Nel listato sopra, vengono innanzitutto eseguiti i comandi di \texttt{list1},
e se l'exit status dell'ultimo comando della lista \`e \texttt{0} (successo),
il blocco passa ad eseguire i comandi in \texttt{list2} e quindi esce; in
caso negativo (ovvero \texttt{list1} fallisce), vengono eseguiti i comandi in
\texttt{list3}, e ancora come prima, se la lista esce con successo vengono
eseguiti \texttt{list4} e quindi il blocco esce, altrimenti vengono eseguiti i
comandi in \texttt{list5}.

\`E importante notare che un blocco if pu\`o contenere zero o pi\`u sezioni
\texttt{elif}, e che la sezione \texttt{else} \`e sempre opzionale.

Un esempio un po' pi\`u lungo per capire meglio il funzionamento:
\begin{verbatim}
#!/bin/bash
stop=0
while [[ $stop -eq 0 ]]
do
        cat << ENDOFMENU
        1: sysadmin
        2: user
        3: guest
ENDOFMENU
echo "Your choice?"
read ch
if [[ "$ch" = "1" ]]; then
        sysadmin
elif [[ "$ch" = "2" ]]; then
        user
elif [[ "$ch" = "3" ]]; then
        guest
else
        echo error
fi
\end{verbatim}

\medskip
Per evitare molteplici utilizzi di \texttt{elif}, che possono risultare
scomodi e rendere meno leggibile il codice, un'altra struttura di controllo
\`e disponibile: il cosiddetto \texttt{case - in - esac}. La struttura \`e la
seguente:
\begin{verbatim}
case expression in
        value1)
                list1
                ;;
        value2)
                list2
                ;;
esac
\end{verbatim}
\texttt{expression} viene valutata, e il suo risultato viene confrontato con i
vari \texttt{value}, dal primo all'ultimo; quando il primo \texttt{value} che
corrisponde al risultato viene trovato, si esegue la \texttt{list}
corrispondente, e finita la sequenza di comandi si salta al \texttt{esac} del
blocco. \`E importante notare che i \texttt{value} possono contenere
wildcards, quindi mettere ad esempio \texttt{*$)$} come ultimo value ha senso
per prendere tutti i risultati che non rispecchiano i valori specificati
sopra.
Ad esempio:
\begin{verbatim}
#!/bin/bash
# Stampa gli utenti che consumano piu` spazio su HD
case "$1" in
  "")
    lines=50
    ;;
  *[!0-9]*)
    echo "Usage: `basename $0` usersnum";
    exit 1
    ;;
  *)
    lines=$1
    ;;
esac
du -s /home/* | sort -gr | head -$lines
\end{verbatim}

\medskip
Per ripetere pi\`u volte una lista di comandi finch\'e una data condizione non
sia vera, la shell ci mette a disposizione tre tipi di comandi:
\begin{itemize}
	\item \texttt{while - do - done}
	\item \texttt{until - do - done}
	\item \texttt{for - in - do - done}
\end{itemize}
Le strutture sono cos\`\i definite:
\begin{verbatim}
while list1
do
        list2
done

until list1
do
        list2
done
\end{verbatim}
per while e until, molto simili fra loro: nel while, \texttt{list1} viene
eseguita, e se l'ultimo comando esce con successo viene eseguita
\texttt{list2}, e si ritorna quindi all'inizio del ciclo con una nuova
esecuzione di \texttt{list1}, finch\'e quest'ultima non esce con status
fallimentare. Nell'until funziona alla stessa maniera, salvo il fatto che il
ciclo viene ripetuto finch\'e \texttt{list1} \`e falsa, e quando invece esce
con successo, il ciclo termina.

A questi due cicli si applicano due istruzioni: \texttt{break}, che causa
l'immediata uscita dal ciclo, e \texttt{continue}, che riporta immediatamente
il ciclo all'inizio.

Ad esempio, possiamo utilizzare un ciclo until e un while per stampare le
tabelline...
\begin{verbatim}
#!/bin/bash
if [ "$1" -eq "" ]; then
  echo "Usage: `basename $0` max"
  exit
fi
x=1
until [ $1 -gt $x ]
do
  y=1
  while [ $y -le $1 ]
  do
    echo `expr $x \* $y` ""
    y=`expr $y + 1`
  done
  echo
  x=`expr $x + 1`
done
\end{verbatim}

Il ciclo for, invece, ha la seguente struttura:
\begin{verbatim}
for name in words
do
        list
done
\end{verbatim}

Ciascuna parola nella lista \texttt{words} viene caricata nella variabile
\texttt{name} ad ogni iterazione del ciclo, e quindi viene eseguita
\texttt{list}. Se la clausola \texttt{in} viene omessa, al suo posto viene
utilizzata la variabile \texttt{\$*}.
Ancora un esempio:
\begin{verbatim}
#!/bin/bash
for color in red yellow blue green
do
  echo One color is $color
done
\end{verbatim}

\section{Comandi built-in}
\subsection{I/O}
\subsubsection*{\texttt{echo}}
Stampa i suoi argomenti su \texttt{stdout}. L'opzione \texttt{-n} viene
utilizzata per evitare che la stampa includa il newline finale. Utilizzato con
command substitution, pu\`o servire per settare una variabile:
\begin{verbatim}
a=`echo "HELLO" | tr A-Z a-z`
\end{verbatim}
Per abilitare l'interpretazione delle sequenze di escape, si specifica
l'opzione \texttt{-e}.

\subsubsection*{\texttt{printf}}
Versione migliorata di \texttt{echo}, la sintassi \`e \emph{simile} a quella
della funzione C \texttt{printf(3)}. L'interpretazione delle sequenze di
escape \`e abilitata di default.

\subsubsection*{\texttt{read \emph{var}}}
Legge l'input da stdin e lo archivia nella variabile specificata
(\texttt{var}). L'opzione \texttt{-p} evita la stampa dell'input,
\texttt{-nN} accetta solo N caratteri in input, e \texttt{-p} stampa una
stringa alla chiamata di \texttt{read}.
\begin{verbatim}
read -s -n1 -p "Premi un tasto..." keypress
echo; echo "Il tasto premuto e` \"$keypress\"."
\end{verbatim}

\subsection{variabili}
\subsubsection*{\texttt{declare}, \texttt{typeset}}
Permettono di restringere le propriet\`a di una variabile, una forma debole di
gestione dei tipi. L'opzione \texttt{-r} dichiara una variabile come
read-only (come \texttt{readonly}, \texttt{-i} come intera, \texttt{-a} come
array e \texttt{-x} esporta la variabile (come \texttt{export}).

\subsubsection*{\texttt{let}}
Esegue operazioni aritmetiche sulle variabili, in molti casi pu\`o funzionare
come una versione semplificata di \texttt{expr}.
\begin{verbatim}
let a=11       # assegna ad a il valore 11
let "a <<= 3"  # shift a sinistra di 3 posizioni
let "a /= 4"   # a viene diviso per 4 e il risultato viene riassegnato
let "a -= 5"   # sottrazione di 5 da a e riassegnamento
\end{verbatim}

\subsubsection*{\texttt{unset}}
Cancella il contenuto di una variabile

\subsection{Script execution}
\subsubsection*{\texttt{source}}
Conosciuto anche come \emph{dot command} (\texttt{.}). Esegue uno script senza
aprire una subshell. Corrisponde alla direttiva \verb_#include_ del
preprocessore C.
\begin{verbatim}
$ source ~/bash_profile
$ . ~/bash_profile
\end{verbatim}

\subsection{Comandi vari}
\subsubsection*{\texttt{true}, \texttt{false}}
Ritornano sempre rispettivamente 0 e 1.

\subsubsection*{\texttt{type $[$\emph{cmd}$]$}}
Stampa un messaggio che indica se \texttt{cmd} \`e una keyword, un comando
built-in oppure un comando esterno.

\subsubsection*{\texttt{shopt $[$\emph{options}$]$}}
Setta alcune opzioni della shell (ad esempio, \texttt{shopts -s cdspell}
permette il mispelling in \texttt{cd}).

\subsubsection*{\texttt{alias}, \texttt{unalias}}
Un \emph{alias} \`e una scorciatoia per abbreviare lunghe sequenze di comandi:
\begin{verbatim}
alias dir="ls -l"
alias rd="rm -r"
unalias dir
\end{verbatim}

\subsubsection*{\texttt{history}}
Visualizza l'elenco degli ultimi comandi eseguiti.

\subsection{Directory stack}
La shell Bash permette di manipolare una pila di directory con alcuni comandi
utili per visite ad albero.

\subsubsection*{\texttt{dirs}}
Stampa il contenuto del directory stack.

\subsubsection*{\texttt{pushd \emph{dirname}}}
Esegue il push della directory \emph{dirname} nello stack; successivamente si
sposta nella directory \emph{dirname} ed esegue \texttt{dirs}.

\subsubsection*{\texttt{popd}}
Esegue il pop dallo stack, si sposta sull'attuale top element ed esegue
\texttt{dirs}.

\subsubsection*{\texttt{\$DIRSTACK}}
Contiene il top element dello stack.

\section{File di configurazione}
\subsection{Inizializzazione}
La shell, all'avvio, fa il parsing di alcuni file di configurazione:

\subsubsection*{/etc/profile}
Settings sistem-wide validi per tutte le shell, non solo Bash.

\subsubsection*{/etc/bashrc}
Configurazione sistem-wide valida unicamente per Bash, contenente funzioni,
alias e modelli di comportamento.

\subsubsection*{\$HOME/.bash\_profile}
Settings specifici per l'utente relativi a Bash (se Bash non trova questo
file, cercher\`a il file \texttt{\$HOME/.profile}.

\subsubsection*{\$HOME/.bashrc}
Configurazione di Bash specifica per l'utente, contiene funzioni e alias.
Viene letto solo se la shell \`e di tipo interattivo o in esecuzione di
script.

\subsection{Terminazione}
\subsubsection*{\$HOME/.bash\_logout}
Al logout, Bash esegue lo script contenuto in questo file.

\section{Comandi esterni}



\appendix
% Appendix 1: GNU Free Documentation License

\chapter{GNU Free Documentation License}
%\label{label_fdl}

 \begin{center}

       Version 1.2, November 2002


 Copyright \copyright 2000,2001,2002  Free Software Foundation, Inc.
 
 \bigskip
 
     51 Franklin St, Fifth Floor, Boston, MA  02110-1301  USA
  
 \bigskip
 
 Everyone is permitted to copy and distribute verbatim copies
 of this license document, but changing it is not allowed.
\end{center}


\begin{center}
{\bf\large Preamble}
\end{center}

The purpose of this License is to make a manual, textbook, or other
functional and useful document "free" in the sense of freedom: to
assure everyone the effective freedom to copy and redistribute it,
with or without modifying it, either commercially or noncommercially.
Secondarily, this License preserves for the author and publisher a way
to get credit for their work, while not being considered responsible
for modifications made by others.

This License is a kind of "copyleft", which means that derivative
works of the document must themselves be free in the same sense.  It
complements the GNU General Public License, which is a copyleft
license designed for free software.

We have designed this License in order to use it for manuals for free
software, because free software needs free documentation: a free
program should come with manuals providing the same freedoms that the
software does.  But this License is not limited to software manuals;
it can be used for any textual work, regardless of subject matter or
whether it is published as a printed book.  We recommend this License
principally for works whose purpose is instruction or reference.


\begin{center}
{\Large\bf 1. APPLICABILITY AND DEFINITIONS}
\addcontentsline{toc}{section}{1. APPLICABILITY AND DEFINITIONS}
\end{center}

This License applies to any manual or other work, in any medium, that
contains a notice placed by the copyright holder saying it can be
distributed under the terms of this License.  Such a notice grants a
world-wide, royalty-free license, unlimited in duration, to use that
work under the conditions stated herein.  The \textbf{"Document"}, below,
refers to any such manual or work.  Any member of the public is a
licensee, and is addressed as \textbf{"you"}.  You accept the license if you
copy, modify or distribute the work in a way requiring permission
under copyright law.

A \textbf{"Modified Version"} of the Document means any work containing the
Document or a portion of it, either copied verbatim, or with
modifications and/or translated into another language.

A \textbf{"Secondary Section"} is a named appendix or a front-matter section of
the Document that deals exclusively with the relationship of the
publishers or authors of the Document to the Document's overall subject
(or to related matters) and contains nothing that could fall directly
within that overall subject.  (Thus, if the Document is in part a
textbook of mathematics, a Secondary Section may not explain any
mathematics.)  The relationship could be a matter of historical
connection with the subject or with related matters, or of legal,
commercial, philosophical, ethical or political position regarding
them.

The \textbf{"Invariant Sections"} are certain Secondary Sections whose titles
are designated, as being those of Invariant Sections, in the notice
that says that the Document is released under this License.  If a
section does not fit the above definition of Secondary then it is not
allowed to be designated as Invariant.  The Document may contain zero
Invariant Sections.  If the Document does not identify any Invariant
Sections then there are none.

The \textbf{"Cover Texts"} are certain short passages of text that are listed,
as Front-Cover Texts or Back-Cover Texts, in the notice that says that
the Document is released under this License.  A Front-Cover Text may
be at most 5 words, and a Back-Cover Text may be at most 25 words.

A \textbf{"Transparent"} copy of the Document means a machine-readable copy,
represented in a format whose specification is available to the
general public, that is suitable for revising the document
straightforwardly with generic text editors or (for images composed of
pixels) generic paint programs or (for drawings) some widely available
drawing editor, and that is suitable for input to text formatters or
for automatic translation to a variety of formats suitable for input
to text formatters.  A copy made in an otherwise Transparent file
format whose markup, or absence of markup, has been arranged to thwart
or discourage subsequent modification by readers is not Transparent.
An image format is not Transparent if used for any substantial amount
of text.  A copy that is not "Transparent" is called \textbf{"Opaque"}.

Examples of suitable formats for Transparent copies include plain
ASCII without markup, Texinfo input format, LaTeX input format, SGML
or XML using a publicly available DTD, and standard-conforming simple
HTML, PostScript or PDF designed for human modification.  Examples of
transparent image formats include PNG, XCF and JPG.  Opaque formats
include proprietary formats that can be read and edited only by
proprietary word processors, SGML or XML for which the DTD and/or
processing tools are not generally available, and the
machine-generated HTML, PostScript or PDF produced by some word
processors for output purposes only.

The \textbf{"Title Page"} means, for a printed book, the title page itself,
plus such following pages as are needed to hold, legibly, the material
this License requires to appear in the title page.  For works in
formats which do not have any title page as such, "Title Page" means
the text near the most prominent appearance of the work's title,
preceding the beginning of the body of the text.

A section \textbf{"Entitled XYZ"} means a named subunit of the Document whose
title either is precisely XYZ or contains XYZ in parentheses following
text that translates XYZ in another language.  (Here XYZ stands for a
specific section name mentioned below, such as \textbf{"Acknowledgements"},
\textbf{"Dedications"}, \textbf{"Endorsements"}, or \textbf{"History"}.)  
To \textbf{"Preserve the Title"}
of such a section when you modify the Document means that it remains a
section "Entitled XYZ" according to this definition.

The Document may include Warranty Disclaimers next to the notice which
states that this License applies to the Document.  These Warranty
Disclaimers are considered to be included by reference in this
License, but only as regards disclaiming warranties: any other
implication that these Warranty Disclaimers may have is void and has
no effect on the meaning of this License.


\begin{center}
{\Large\bf 2. VERBATIM COPYING}
\addcontentsline{toc}{section}{2. VERBATIM COPYING}
\end{center}

You may copy and distribute the Document in any medium, either
commercially or noncommercially, provided that this License, the
copyright notices, and the license notice saying this License applies
to the Document are reproduced in all copies, and that you add no other
conditions whatsoever to those of this License.  You may not use
technical measures to obstruct or control the reading or further
copying of the copies you make or distribute.  However, you may accept
compensation in exchange for copies.  If you distribute a large enough
number of copies you must also follow the conditions in section 3.

You may also lend copies, under the same conditions stated above, and
you may publicly display copies.


\begin{center}
{\Large\bf 3. COPYING IN QUANTITY}
\addcontentsline{toc}{section}{3. COPYING IN QUANTITY}
\end{center}


If you publish printed copies (or copies in media that commonly have
printed covers) of the Document, numbering more than 100, and the
Document's license notice requires Cover Texts, you must enclose the
copies in covers that carry, clearly and legibly, all these Cover
Texts: Front-Cover Texts on the front cover, and Back-Cover Texts on
the back cover.  Both covers must also clearly and legibly identify
you as the publisher of these copies.  The front cover must present
the full title with all words of the title equally prominent and
visible.  You may add other material on the covers in addition.
Copying with changes limited to the covers, as long as they preserve
the title of the Document and satisfy these conditions, can be treated
as verbatim copying in other respects.

If the required texts for either cover are too voluminous to fit
legibly, you should put the first ones listed (as many as fit
reasonably) on the actual cover, and continue the rest onto adjacent
pages.

If you publish or distribute Opaque copies of the Document numbering
more than 100, you must either include a machine-readable Transparent
copy along with each Opaque copy, or state in or with each Opaque copy
a computer-network location from which the general network-using
public has access to download using public-standard network protocols
a complete Transparent copy of the Document, free of added material.
If you use the latter option, you must take reasonably prudent steps,
when you begin distribution of Opaque copies in quantity, to ensure
that this Transparent copy will remain thus accessible at the stated
location until at least one year after the last time you distribute an
Opaque copy (directly or through your agents or retailers) of that
edition to the public.

It is requested, but not required, that you contact the authors of the
Document well before redistributing any large number of copies, to give
them a chance to provide you with an updated version of the Document.


\begin{center}
{\Large\bf 4. MODIFICATIONS}
\addcontentsline{toc}{section}{4. MODIFICATIONS}
\end{center}

You may copy and distribute a Modified Version of the Document under
the conditions of sections 2 and 3 above, provided that you release
the Modified Version under precisely this License, with the Modified
Version filling the role of the Document, thus licensing distribution
and modification of the Modified Version to whoever possesses a copy
of it.  In addition, you must do these things in the Modified Version:

\begin{itemize}
\item[A.] 
   Use in the Title Page (and on the covers, if any) a title distinct
   from that of the Document, and from those of previous versions
   (which should, if there were any, be listed in the History section
   of the Document).  You may use the same title as a previous version
   if the original publisher of that version gives permission.
   
\item[B.]
   List on the Title Page, as authors, one or more persons or entities
   responsible for authorship of the modifications in the Modified
   Version, together with at least five of the principal authors of the
   Document (all of its principal authors, if it has fewer than five),
   unless they release you from this requirement.
   
\item[C.]
   State on the Title page the name of the publisher of the
   Modified Version, as the publisher.
   
\item[D.]
   Preserve all the copyright notices of the Document.
   
\item[E.]
   Add an appropriate copyright notice for your modifications
   adjacent to the other copyright notices.
   
\item[F.]
   Include, immediately after the copyright notices, a license notice
   giving the public permission to use the Modified Version under the
   terms of this License, in the form shown in the Addendum below.
   
\item[G.]
   Preserve in that license notice the full lists of Invariant Sections
   and required Cover Texts given in the Document's license notice.
   
\item[H.]
   Include an unaltered copy of this License.
   
\item[I.]
   Preserve the section Entitled "History", Preserve its Title, and add
   to it an item stating at least the title, year, new authors, and
   publisher of the Modified Version as given on the Title Page.  If
   there is no section Entitled "History" in the Document, create one
   stating the title, year, authors, and publisher of the Document as
   given on its Title Page, then add an item describing the Modified
   Version as stated in the previous sentence.
   
\item[J.]
   Preserve the network location, if any, given in the Document for
   public access to a Transparent copy of the Document, and likewise
   the network locations given in the Document for previous versions
   it was based on.  These may be placed in the "History" section.
   You may omit a network location for a work that was published at
   least four years before the Document itself, or if the original
   publisher of the version it refers to gives permission.
   
\item[K.]
   For any section Entitled "Acknowledgements" or "Dedications",
   Preserve the Title of the section, and preserve in the section all
   the substance and tone of each of the contributor acknowledgements
   and/or dedications given therein.
   
\item[L.]
   Preserve all the Invariant Sections of the Document,
   unaltered in their text and in their titles.  Section numbers
   or the equivalent are not considered part of the section titles.
   
\item[M.]
   Delete any section Entitled "Endorsements".  Such a section
   may not be included in the Modified Version.
   
\item[N.]
   Do not retitle any existing section to be Entitled "Endorsements"
   or to conflict in title with any Invariant Section.
   
\item[O.]
   Preserve any Warranty Disclaimers.
\end{itemize}

If the Modified Version includes new front-matter sections or
appendices that qualify as Secondary Sections and contain no material
copied from the Document, you may at your option designate some or all
of these sections as invariant.  To do this, add their titles to the
list of Invariant Sections in the Modified Version's license notice.
These titles must be distinct from any other section titles.

You may add a section Entitled "Endorsements", provided it contains
nothing but endorsements of your Modified Version by various
parties--for example, statements of peer review or that the text has
been approved by an organization as the authoritative definition of a
standard.

You may add a passage of up to five words as a Front-Cover Text, and a
passage of up to 25 words as a Back-Cover Text, to the end of the list
of Cover Texts in the Modified Version.  Only one passage of
Front-Cover Text and one of Back-Cover Text may be added by (or
through arrangements made by) any one entity.  If the Document already
includes a cover text for the same cover, previously added by you or
by arrangement made by the same entity you are acting on behalf of,
you may not add another; but you may replace the old one, on explicit
permission from the previous publisher that added the old one.

The author(s) and publisher(s) of the Document do not by this License
give permission to use their names for publicity for or to assert or
imply endorsement of any Modified Version.


\begin{center}
{\Large\bf 5. COMBINING DOCUMENTS}
\addcontentsline{toc}{section}{5. COMBINING DOCUMENTS}
\end{center}


You may combine the Document with other documents released under this
License, under the terms defined in section 4 above for modified
versions, provided that you include in the combination all of the
Invariant Sections of all of the original documents, unmodified, and
list them all as Invariant Sections of your combined work in its
license notice, and that you preserve all their Warranty Disclaimers.

The combined work need only contain one copy of this License, and
multiple identical Invariant Sections may be replaced with a single
copy.  If there are multiple Invariant Sections with the same name but
different contents, make the title of each such section unique by
adding at the end of it, in parentheses, the name of the original
author or publisher of that section if known, or else a unique number.
Make the same adjustment to the section titles in the list of
Invariant Sections in the license notice of the combined work.

In the combination, you must combine any sections Entitled "History"
in the various original documents, forming one section Entitled
"History"; likewise combine any sections Entitled "Acknowledgements",
and any sections Entitled "Dedications".  You must delete all sections
Entitled "Endorsements".

\begin{center}
{\Large\bf 6. COLLECTIONS OF DOCUMENTS}
\addcontentsline{toc}{section}{6. COLLECTIONS OF DOCUMENTS}
\end{center}

You may make a collection consisting of the Document and other documents
released under this License, and replace the individual copies of this
License in the various documents with a single copy that is included in
the collection, provided that you follow the rules of this License for
verbatim copying of each of the documents in all other respects.

You may extract a single document from such a collection, and distribute
it individually under this License, provided you insert a copy of this
License into the extracted document, and follow this License in all
other respects regarding verbatim copying of that document.


\begin{center}
{\Large\bf 7. AGGREGATION WITH INDEPENDENT WORKS}
\addcontentsline{toc}{section}{7. AGGREGATION WITH INDEPENDENT WORKS}
\end{center}


A compilation of the Document or its derivatives with other separate
and independent documents or works, in or on a volume of a storage or
distribution medium, is called an "aggregate" if the copyright
resulting from the compilation is not used to limit the legal rights
of the compilation's users beyond what the individual works permit.
When the Document is included in an aggregate, this License does not
apply to the other works in the aggregate which are not themselves
derivative works of the Document.

If the Cover Text requirement of section 3 is applicable to these
copies of the Document, then if the Document is less than one half of
the entire aggregate, the Document's Cover Texts may be placed on
covers that bracket the Document within the aggregate, or the
electronic equivalent of covers if the Document is in electronic form.
Otherwise they must appear on printed covers that bracket the whole
aggregate.


\begin{center}
{\Large\bf 8. TRANSLATION}
\addcontentsline{toc}{section}{8. TRANSLATION}
\end{center}


Translation is considered a kind of modification, so you may
distribute translations of the Document under the terms of section 4.
Replacing Invariant Sections with translations requires special
permission from their copyright holders, but you may include
translations of some or all Invariant Sections in addition to the
original versions of these Invariant Sections.  You may include a
translation of this License, and all the license notices in the
Document, and any Warranty Disclaimers, provided that you also include
the original English version of this License and the original versions
of those notices and disclaimers.  In case of a disagreement between
the translation and the original version of this License or a notice
or disclaimer, the original version will prevail.

If a section in the Document is Entitled "Acknowledgements",
"Dedications", or "History", the requirement (section 4) to Preserve
its Title (section 1) will typically require changing the actual
title.


\begin{center}
{\Large\bf 9. TERMINATION}
\addcontentsline{toc}{section}{9. TERMINATION}
\end{center}


You may not copy, modify, sublicense, or distribute the Document except
as expressly provided for under this License.  Any other attempt to
copy, modify, sublicense or distribute the Document is void, and will
automatically terminate your rights under this License.  However,
parties who have received copies, or rights, from you under this
License will not have their licenses terminated so long as such
parties remain in full compliance.


\begin{center}
{\Large\bf 10. FUTURE REVISIONS OF THIS LICENSE}
\addcontentsline{toc}{section}{10. FUTURE REVISIONS OF THIS LICENSE}
\end{center}


The Free Software Foundation may publish new, revised versions
of the GNU Free Documentation License from time to time.  Such new
versions will be similar in spirit to the present version, but may
differ in detail to address new problems or concerns.  See
http://www.gnu.org/copyleft/.

Each version of the License is given a distinguishing version number.
If the Document specifies that a particular numbered version of this
License "or any later version" applies to it, you have the option of
following the terms and conditions either of that specified version or
of any later version that has been published (not as a draft) by the
Free Software Foundation.  If the Document does not specify a version
number of this License, you may choose any version ever published (not
as a draft) by the Free Software Foundation.


\begin{center}
{\Large\bf ADDENDUM: How to use this License for your documents}
\addcontentsline{toc}{section}{ADDENDUM: How to use this License for your documents}
\end{center}

To use this License in a document you have written, include a copy of
the License in the document and put the following copyright and
license notices just after the title page:

\bigskip
\begin{quote}
    Copyright \copyright  YEAR  YOUR NAME.
    Permission is granted to copy, distribute and/or modify this document
    under the terms of the GNU Free Documentation License, Version 1.2
    or any later version published by the Free Software Foundation;
    with no Invariant Sections, no Front-Cover Texts, and no Back-Cover Texts.
    A copy of the license is included in the section entitled "GNU
    Free Documentation License".
\end{quote}
\bigskip
    
If you have Invariant Sections, Front-Cover Texts and Back-Cover Texts,
replace the "with...Texts." line with this:

\bigskip
\begin{quote}
    with the Invariant Sections being LIST THEIR TITLES, with the
    Front-Cover Texts being LIST, and with the Back-Cover Texts being LIST.
\end{quote}
\bigskip
    
If you have Invariant Sections without Cover Texts, or some other
combination of the three, merge those two alternatives to suit the
situation.

If your document contains nontrivial examples of program code, we
recommend releasing these examples in parallel under your choice of
free software license, such as the GNU General Public License,
to permit their use in free software.

%---------------------------------------------------------------------


\end{document}

