\documentclass[a4paper,10pt]{report}
\pagestyle{headings}

\title{\textsc{\textbf{Introduzione alla shell}}}

\author{Ciro Mattia Gonano <ciro@winged.it>}

\date{}

\usepackage[italian]{babel}
\usepackage[T1]{fontenc}
\frenchspacing
%\usepackage{longtable}

\begin{document}

\maketitle

\section{Disclaimer}

Copyright (c) 2003-2006 Ciro Mattia Gonano.
Permission is granted to copy, distribute and/or modify this document under
the terms of the GNU Free Documentation License, Version 1.2 or any later
version published by the Free Software Foundation; with no Invariant Sections,
no Front-Cover Texts, and no Back-Cover Texts.

A copy of the license is included in the section entitled "GNU Free
Documentation License".

\medskip
Vi sarei grato se riportaste qualsiasi errore, imprecisione, difficolt\`a
trovaste nel testo. Contribuirete all'evoluzione della presente guida, e molta
altra gente ve ne sar\`a grata ;-)

\section{Ringraziamenti}
La prima stesura del presente documento \`e avvenuta prendendo spunto per la
sua quasi totale integrit\`a dai lucidi del corso di ``Laboratorio di Sistemi
Operativi'' del professor \textbf{Alberto Montresor}.

\section{Nota alla presente versione}

Questa versione (generata in data \date) non ha subito cambiamenti da due anni.

Questo implica che, come in ogni cosa che riguardi l'informatica, diverse cose
potrebbero essere cambiate o migliorate; inoltre la presente guida \`e tutto
fuorch\'e finita.

Ho deciso di pubblicarla unicamente per dare un ``la'' a chi fosse interessato
ad approfondire l'utilizzo della shell, ed eventualmente come spunto per chi
volesse contribuire ad ampliare e completare la guida stessa.

Ovviamente i sorgenti sono disponibili su richiesta.

\tableofcontents

% Chapter 1: Introduzione e note tipografiche
\chapter{Introduzione}

\section{Cos'\`e la shell di Linux}
La shell di Linux costituisce l'interfaccia tra il sistema e l'utente umano.

Consente di:
\begin{itemize}
	\item poter interagire con il sistema a svariati livelli
	\item richiamare altri programmi
	\item programmare degli script\footnote{gli script sono programmi in
		formato testuale formati da sequenze di comandi, fondamentali
		per ottimizzare e semplificare la vita all'amministratore del
		sistema}
\end{itemize}

In ambiente Linux abbiamo a disposizione svariati tipi di shells: \emph{sh, ash,
csh, bash, esh, lsh, kiss, pdksh, sash, psh, tcsh, zsh} tanto per citarne solo
alcune. Alcune sono derivate (e potenziate) da altre\footnote{come \emph{bash} \`e
una versione potenziata di \emph{sh}}, mentre altre differiscono in maniera
notevole, sia nell'uso che nell'impostazione verso l'utente.

Per la grande variet\`a presente, qui ci concentreremo sull'utilizzo di
\emph{bash}, che \`e ormai lo standard "de facto" sui sistemi
Linux\footnote{attenzione: NON Unix/BSD, per i quali lo standard \`e in genere
\emph{csh} o una sua derivata}.

\section{Prerequisiti}
Lo shell scripting non \`e materia astrusa, ma nemmeno banale. Nel corso di
queste pagine daremo per scontate alcune nozioni che riteniamo fondamentali
per un utente Linux, su cui non ci soffermeremo per non appesantire il
documento.
In particolare daremo per scontata la conoscenza di:
\begin{itemize}
	\item nozioni di login/logout;
	\item organizzazione del filesystem (file e directory, gerarchie,
	pathname assoluti e relativi);
	\item comandi di base:
	\begin{itemize}
		\item listare il contenuto di una directory (\texttt{ls});
		\item maneggiare i file (\texttt{cp}, \texttt{mv}, \texttt{rm}, 
			\texttt{mkdir}, \texttt{rmdir});
		\item visualizzare il contenuto di un file
		(\texttt{cat/more/less/head/tail});
	\end{itemize}
	\item attributi dei file (\texttt{read/write/execute})
	\item conoscenza del significato di permesso (\texttt{chmod})
	\item gestione dei gruppi e dei proprietari dei file (\texttt{chgrp/chown})
\end{itemize}
per cui, se non vi sentite sicuri di questi concetti, vi consigliamo di
rivederli prima di avvicinarvi al bash scripting.

\section{Note tipografiche}
Per facilitare la lettura e la comprensione del testo, sono state utilizzate
le seguenti convenzioni tipografiche:
\begin{itemize}
	\item lo stile {\ttfamily macchina da scrivere} \`e utilizzato per i
		comandi, le variabili, le parole chiave e gli script;
	\item in {\itshape corsivo} le definizioni di nuovi termini,
		gli argomenti dei comandi e i nomi dei file;
%	\item lo stile {\sffamily sans serif}
\end{itemize}



% Chapter 2: Nozioni di base
\chapter{Nozioni di base}

\section{Caratteristiche della shell}
Ogni shell (non solo bash) presenta una serie di caratteristiche, raggruppabili in macrocategorie:
\begin{itemize}
	\item comandi built-in;
	\item metacaratteri;
	\item redirezione di I/O;
	\item command substitution;
	\item sequenze condizionali e non;
	\item raggruppamento di comandi;
	\item esecuzione in background;
	\item quoting;
	\item subshell;
	\item variabili (locali/d'ambiente);
	\item scripting.
\end{itemize}
Passiamo a vederli in generale.

\section{Comandi built-in}
Quando richiedete l'esecuzione di un comando esterno, la shell si
occupa di richiamare il corrispondente file eseguibile, caricarlo in memoria
ed eseguirlo (ad esempio, quando chiamiamo il comando \texttt{ls},
\textit{bash} si occupa di trovare e lanciare \texttt{/bin/ls}).

Al contrario, quando richiedete un comando che possa essere riconosciuto ed
eseguito direttamente dalla shell, fate uso di un comando \textit{interno},
detto anche \textit{built-in}.\\
Esempi:
\begin{verbatim}
  echo         # stampa tutti i suoi argomenti sullo standard output
  cd           # cambia directory
\end{verbatim}

\section{Metacaratteri}
La shell dispone di una serie di caratteri che non sono considerati nella loro
forma a video, ma servono a specificare serie di funzionalit\`a interne alla
shell. Per questo motivo quando volete digitare uno di tali caratteri dovete
solitamente farne l'\textit{escape} con il carattere \textit{backslash}
(``$\backslash$'').

I metacaratteri possono essere trattati in maniera leggermente diversa a
seconda della shell utilizzata, ma in generale sono questi\footnote{e
\textit{bash} segue questo elenco}:
\begin{itemize}
	\item \texttt{>, >>, <\ \ \ \ \ \ \ \ } redirezione I/O
	\item \texttt{|\ \ \ \ \ \ \ \ \ \ \ \ \ \ } pipe
	\item \texttt{*, ?, [\ldots]\ \ \ \ } wildcards
	\item \texttt{`command`\ \ \ \ \ \ } command substitution
	\item \texttt{;\ \ \ \ \ \ \ \ \ \ \ \ \ \ } esecuzione sequenziale
	\item \texttt{||, \&\&\ \ \ \ \ \ \ \ \ } esecuzione condizionale
	\item \texttt{(\ldots)\ \ \ \ \ \ \ \ \ \ } raggruppamento comandi
	\item \texttt{\&\ \ \ \ \ \ \ \ \ \ \ \ \ \ } esecuzione in background
	\item \texttt{"", '{}'\ \ \ \ \ \ \ \ \ \ } quoting
	\item \texttt{\#\ \ \ \ \ \ \ \ \ \ \ \ \ \ } commento
	\item \texttt{\$\ \ \ \ \ \ \ \ \ \ \ \ \ \ } espansione di variabile
	\item \texttt{$\backslash$\ \ \ \ \ \ \ \ \ \ \ \ \ \ } carattere di backslash
	\item \texttt{<<\ \ \ \ \ \ \ \ \ \ \ \ \ \ } ``here documents''
\end{itemize}

\section{Redirezione I/O}
Ogni processo \`e associato a tre ``stream'': lo \textit{standard input}
(\texttt{stdin}), lo \textit{standard output} (\texttt{stdout}) e lo
\textit{standard error} (\texttt{stderr}).

La redirezione dell'I/O permette di associare questi stream a
sorgenti/destinazioni\footnote{ovviamente nel caso di \texttt{stdin} avremo
una sorgente, mentre per \texttt{stdout} e \texttt{stderr} avremo due
destinazioni -- che \emph{possono} coincidere} diverse dalle loro attuali.
Possiamo in questo modo salvare l'output di un processo in un file di testo,
oppure possiamo salvare il suo log di errore, attraverso semplici comandi si
questo tipo:
\begin{verbatim}
  ls > list.txt     # salva l'output di ls in un file list.txt
  ls >> list.txt    # aggiunge l'output di ls in coda
                    # al file list.txt
  rm fileinutile >& /dev/null  # redireziona stdout e stderr
                               # di rm su /dev/null
\end{verbatim}
Notate che, se non specificato, ``>'' ridirige lo \texttt{stdout}, e ``<'' lo
\texttt{stdin}. ``>\&'' ridirige \emph{tutti} i flussi di output (quindi anche
lo \texttt{stderr}). Notate anche il diverso significato di ``>'' e ``>>'': il
primo se rediretto in un file lo tronca al suo inizio, cancellando il suo
contenuto prima di scrivere il flusso, mentre il secondo conserva il contenuto
del file e accoda ad esso il flusso.

\addcontentsline{toc}{subsection}{Pipe, o catene di montaggio}
\subsection*{Pipe, o catene di montaggio}
Attraverso le pipe, potete utilizzare l'output di un processo come input di un
altro processo:
\begin{verbatim}
$ ls
  a.c b.c a.o b.o dir1 dir2
$ ls | wc -w
  6
\end{verbatim}
\texttt{wc} (un programma che conta parole, linee e molto altro) prende come
\texttt{stdin} lo \texttt{stdout} di \texttt{ls}, ritornando il numero di
parole listate da quest'ultimo: 6, appunto.

\medskip
Potrebbe sembrare una grossa limitazione poter unicamente ridirigere il
flusso, e non poterlo duplicare: a questo supplisce il comando \texttt{tee}
\footnote{il nome deriva dai ``giunti a T'', utilizzati dagli idraulici}
che copia lo \texttt{stdin} sul file specificato E sullo \texttt{stdout}:
\begin{verbatim}
$ who | tee who_capture.txt | sort
\end{verbatim}
Se \texttt{tee} viene chiamato con l'opzione \texttt{-a}, viene fatto l'append
al file argomento, invece del troncamento.

\section{Wildcards}
Le wildcards sono utilizzate per specificare pattern comprendenti pi\`u file:
i file vengono passati in rassegna, e quelli che corrispondono\footnote{in
gergo, ``\emph{to match}''} vengono restituiti. Vengono utlizzati pi\`u spesso
delle \textit{regular expressions} per la loro semplicit\`a, ad ogni modo sono
un sottoinsieme di queste ultime.
\begin{itemize}
	\item \texttt{*\ \ \ \ \ } match di qualsiasi stringa (anche vuota)
	\item \texttt{?\ \ \ \ \ } match di qualsiasi carattere singolo
	\item \texttt{[\ldots]\ } match di qualsiasi carattere inserito nelle
		parentesi
\end{itemize}
Esempi:
\begin{verbatim}
$ ls *.c
 prova001.c prova004.c prova125.c prov.c
$ ls prova00?.c
 prova001.c prova004.c
$ ls prova[0-9][0-9][0-9].c
 prova001.c prova004.c prova125.c
\end{verbatim}

\section{Command substitution}
Si pu\`o sostituire ad un argomento un comando del tipo
\texttt{`comando`}: il suo standard output verr\`a sostituito al comando
stesso:
\begin{verbatim}
$ echo Data odierna: `date`
$ echo Utenti collegati: `who | wc -l`
$ tar czvf src-`date`.tar.gz src/
\end{verbatim}

\section{Sequenze}
\addcontentsline{toc}{subsection}{Non condizionali}
\subsection*{Non condizionali}
Si possono eseguire sequenzialmente dei comandi utilizzando il metacarattere
``;'':
\begin{verbatim}
$ date ; pwd ; ls
\end{verbatim}

\addcontentsline{toc}{subsection}{Condizionali}
\subsection*{Condizionali}
Le sequenze condizionali specificano la sequenza da eseguire in base
all'\emph{exit code} del comando precedente all'interno della sequenza.

``||'' viene utilizzato per eseguire il comando nel caso l'\emph{exit code}
del precedente sia diverso da 0 (il che indica un fallimento).

``\&\&'' invece esegue il comando se il precedente esce con un \emph{exit
code} uguale a 0 (cio\`e esce con successo).

Esempi:
\begin{verbatim}
$ gcc prova.c && ./a.out
$ gcc prova.c || echo Compilazione fallita
\end{verbatim}

\section{Raggruppamento di comandi}
\label{sec:sub:raggr}
\`E possibile raggruppare dei comandi all'interno di ``('' e ``)'': verranno
eseguiti in una subshell, e condivideranno gli stessi \texttt{stdin},
\texttt{stdout} e \texttt{stderr}:
\begin{verbatim}
$ date ; ls ; pwd > out.txt
$ (date ; ls ; pwd) > out.txt
\end{verbatim}

\section{Esecuzione in background}
\label{sec:sub:backg}
Se il comando lanciato da shell viene seguito dal metacarattere ``\&'', viene
creata una subshell, il comando viene avviato in background e il controllo
torna alla shell immediatamente, che prosegue l'esecuzione concorrentemente
con il processo lanciato. Quest'ultimo abbandoner\`a la sorgente di
\texttt{stdin} come tastiera (quindi non sar\`a possibile fornirgli input in
tale modo), ritornadolo alla shell che vi permetter\`a di proseguire il
lavoro; ovviamente \`e una funzionalit\`a molto utile se dovete eseguire
attivit\`a lunghe che non necessitano il vostro input:
\begin{verbatim}
$ find / -name passwd -print &
 [34256]
$ /etc/passwd
$ find / -name passwd -print &> results.txt &
 [34263]
$
\end{verbatim}

\section{Quoting}
Esiste la possibilit\`a di disabilitare \emph{command substitution},
\emph{wildcards} e \emph{variable substitution} (per non essere costretti a
fare necessariamente l'escape di ogni carattere) attraverso il \emph{quoting}.

Con gli apici singoli (\texttt{'{}'}) vengono inibiti \emph{wildcards},
\emph{command substitution} e \emph{variable substitution}:
\begin{verbatim}
$ echo 3 * 4 = 12
$ echo '3 * 4 = 12'
\end{verbatim}

Con i doppi apici (\texttt{""}), invece, vengono disabilitate le sole
\emph{wildcards}:
\begin{verbatim}
$ export nome="mionome"
$ echo "Il mio nome: $nome - la data: `date`"
$ echo 'Il mio nome: $nome - la data: `date`'
\end{verbatim}

\section{Subshell}
Quando aprite un terminale (ad esempio quando fate login) viene eseguita una
\emph{shell}. Viene creata una \emph{child shell} (o \emph{subshell}) quando
\begin{itemize}
	\item usate il raggruppamento di comandi (vedi \ref{sec:sub:raggr})
	\item eseguite un processo in background (vedi \ref{sec:sub:backg})
	\item eseguite uno script
\end{itemize}
Le \emph{subshell} hanno la propria directory corrente, la propria area di
variabili indipendenti dalla \emph{shell} madre (vedi \ref{sec:sub:var} per le
variabili).

\section{Variabili}
\label{sec:sub:var}
Le variabili supportate da qualsiasi shell sono di due tipi: \emph{globali} e
\emph{locali}.

Le variabili \emph{locali} non vengono passate da una shell alle altre
subshell.

Al contrario, le variabili \emph{globali} (chiamate anche variabili
\emph{d'ambiente}) vengono passate da una shell alle subshell che essa crea, e
vengono utilizzate per comunicazioni tra shell madre e figlie.

Entrambi i tipi di variabile contengono dati di tipo \textit{stringa}. Ogni
shell ha una serie di variabili d'ambiente che vengono inizializzate all'avvio
della shell stessa (in genere attraverso i file \textit{/etc/profile},
\textit{/home/nomeutente/.bash\_profile}, etc.), come:
\begin{verbatim}
$HOME, $PATH, $USER, $SHELL, $TERM, ...
\end{verbatim}
Un elenco completo delle variabili attualmente in uso si ottiene attraverso il
comando \texttt{env}.

Per accedere al contenuto di una variabile, si pu\`o utilizzare il comando \$;
\$VARIABILE \`e la forma abbreviata di \$\{VARIABILE\} (a volte quest'ultima
forma \`e necessaria):
\begin{verbatim}
$ echo $SHELL
 /bin/bash
\end{verbatim}

Per assegnare un valore ad una variabile, nel caso si tratti di una variabile
locale, basta il nome della variabile seguito dal valore\footnote{fate
attenzione, perch\'e variabile e valore devono essere inframezzate da un segno
di uguale (=), ma tra il segno e i caratteri \textbf{non ci devono essere
spazi}}:
\begin{verbatim}
$ mionome=Ciro\ Mattia\ Gonano   # problemi con gli spazi
$ mionome="Ciro Mattia Gonano"   # nessun problema ;)
\end{verbatim}
Per trasformare una variabile locale dichiarata in questo modo, utilizzare il
comando \texttt{export}\footnote{\texttt{export} funziona solo se usate una
shell di tipo \emph{bourne} (come \emph{bash}), mentre se utilizzate una
\emph{c shell} (come \emph{csh}) dovete usare il comando \texttt{setenv}}:
\begin{verbatim}
$ mionome="Ciro Mattia Gonano"
$ export mionome
\end{verbatim}

Un esempio pi\`u lungo per capire come funzionano le variabili locali e
d'ambiente:
\begin{verbatim}
$ nome="Ciro Mattia"
$ cognome="Gonano"
$ echo $nome $cognome
$ export nome
$ bash
$ echo $nome $cognome
$ exit
$ echo $nome $cognome
\end{verbatim}

\section{Scripting}
Uno script \`e una sequenza di comandi che devono essere eseguiti (e,
ovviamente, pu\`o contenere anche sequenze condizionali), memorizzata
all'interno di un file di testo e poi richiamato dalla shell.
Risulta molto utile per automatizzare procedure lunghe che ripetete
frequentemente.

Per scrivere uno script, \`e sufficiente scrivere la sequenza di comandi nel
file di testo, renderlo eseguibile (con \texttt{chmod}), e
lanciarlo\footnote{state attenti al PATH: se la directory corrente non \`e
all'interno della vostra variabile \texttt{\$PATH}, dovrete lanciare qualcosa
di simile a \texttt{./mioscript}}; la shell si occuper\`a di decidere la shell
con cui eseguire lo script, di creare la subshell, e di passare il contenuto
del file come \texttt{stdin} della subshell.

La selezione della shell da usare per lo script \`e basata sulla prima riga
dello stesso script:
\begin{itemize}
	\item se contiene solo un simbolo \texttt{\#}, verr\`a utilizzata la
		shell corrente;
	\item se contiene \texttt{\#!\textit{pathname}}, verr\`a utilizzata la
		shell identificata da \textit{pathname}: per eseguire uno
		script con la Korn shell scriveremo nella prima riga dello
		script \texttt{\#!/bin/ksh}. Questa \`e la forma che vi
		consiglio di utilizzare sempre, perch\'e risulta non ambigua e
		non dipende dalla configurazione su cui lo script viene
		lanciato.
	\item se non contiene nessuno di questi due, viene interpretato da una
		Bourne shell (\emph{bash}, \emph{sh}, etc.).
\end{itemize}
Alcuni comportamenti interessanti per farvi capire come viene eseguito uno
script\footnote{eseguiteli senza paura, non causeranno danni al vostro
sistema, se li eseguite da normale utente}...
\begin{verbatim}
#!/bin/cat
#!/bin/rm
\end{verbatim}

Esistono alcune variabili \emph{built-in} studiate apposta per gli script (di
seguito sono indicate se valgono per le Bourne shell (\textit{sh}), per le C
shell (\textit{csh}) o per entrambe):
\begin{itemize}
	\item \texttt{\$\$\ \ \ \ \ \ } l'identificatore di processo della
		shell (\textit{sh})
	\item \texttt{\$0\ \ \ \ \ \ } il nome dello shell script
		(\textit{sh,csh})
	\item \texttt{\$1-\$9\ \ \ } l'n-esimo argomento della riga di comando
		(\textit{sh,csh})
	\item \texttt{\$\{n\}\ \ \ \ } l'n-esimo argomento della riga di
		comando (\textit{sh,csh})
	\item \texttt{\$*\ \ \ \ \ \ } lista di tutti gli argomenti della riga
		di comando (\textit{sh,csh})
	\item \texttt{\$\#\ \ \ \ \ \ } numero degli argomenti della riga di
		comando (\textit{sh})
	\item \texttt{\$?\ \ \ \ \ \ } valore di uscita dell'ultimo comando
		eseguito (\textit{sh})
\end{itemize}

Un piccolo script per cominciare a capire...
\begin{verbatim}
#!/bin/bash
a=23
echo $a
b=$a
echo $b
# Questo e` un commento!
# Proviamo qualcosa di piu` eccitante!
a=`echo Ciao mondo\!` # assegna il risultato di echo ad a
echo $a
a=`ls -l`  # adesso a e` uguale all'output
           # di ls -l
echo $a
exit 0
\end{verbatim}

\section{Here document}
L'ultimo metacarattere che ci rimane da vedere \`e il cosiddetto ``Here
document'' (``\texttt{<<}''). I comandi
\begin{verbatim}
<comando> << <parola>

<comando> <</ <parola>
\end{verbatim}
servono a copiare il proprio \texttt{stdin} fino alla linea che inizia con
\texttt{<parola>} (esclusa), e quindi eseguire \texttt{<command>} utilizzando
questi dati copiati come \texttt{stdin}.

La versione con la \textit{slash} (``\texttt{<</}'') non esegue variable
substitution.

Esempio:
\begin{verbatim}
$ cat << ENDOFTEXT > nota
> Ricordarsi di specificare che
> con << le variabili vengono sostituite
> e con <</ no.
>                  $mionome
> ENDOFTEXT
$ 
\end{verbatim}



% Chapter 3: Programmazione della shell
\chapter{Programmazione della shell}

\section{Valutazione delle espressioni}
La shell non supporta direttamente (come faceva, per chi lo ricorda, il caro vecchio
\textbf{C$=$64}, avendo il \emph{BASIC} integrato) la valutazione delle
espressioni; a questo supplisce il comando \texttt{expr \textit{espressione}},
che ci ritorna il risultato della formula introdotta. Tutti i componenti
dell'espressione devono essere necessariamente separati da spazi tra di loro,
e tutti i metacaratteri devono essere ovviamente "escaped"\footnote{come
dicevamo prima, apponendo una \textit{backslash} ($\backslash$) al carattere
(ad esempio: $\backslash$*)}, altrimenti la shell li espander\`a secondo il
loro metasignificato.

Il risultato pu\`o essere numerico o una stringa, e pu\`o venire assegnato ad
una variabile con un uso opportuno del \emph{command substitution}.

Gli operatori disponibili sono:
\begin{itemize}
	\item Aritmetici:\texttt{\ \ \ \ + - * / \%}
	\item Confronto:\texttt{\ \ \ \ = != > < >= <=}
	\item Logici:\texttt{\ \ \ \ \ \ \ \ \& | !}
	\item Parentesi\footnote{devono essere prefissate dalla
	\emph{backslash}}: \texttt{\ \ \ ( )}
	\item Stringhe:
	\begin{itemize}
		\item \texttt{match \textit{string regexp}}
		\item \texttt{substr \textit{string start length}}
		\item \texttt{length \textit{string}}
		\item \texttt{index \textit{string charList}}
	\end{itemize}
\end{itemize}

\section{Exit status}
Ogni comando lanciato dalla shell ritorna un \emph{exit status}. Per
convenzione UNIX, un \emph{exit status} pari a 0 indica ``uscita con
successo'' (pari al valore \texttt{TRUE} nella valutazione delle espressioni
booleane), mentre un valore diverso indica ``uscita (spesso prematura) con
fallimento''. L'\emph{exit status} torna molto utile negli script per
controllare il flusso dell'esecuzione dei comandi, modificandolo in base ai
risutlati d'uscita prodotti. Proprio per questo, la shell ci offre il comando
\texttt{exit \textit{nn}} per far terminare il nostro script con \emph{exit
status} pari a \textit{nn}, e la variabile \emph{built-in} \texttt{\$?}, che
contiene l'\emph{exit status} dell'ultimo comando eseguito:
\begin{verbatim}
$ cat script.sh
#!/bin/bash
echo hello
echo $?     # "echo hello" ha tornato exit status 0 (successo)
lskdf       # comando inesistente
echo $?     # "lskdf" torna exit status diverso da 0 (fallimento)
exit 113    # il nostro script esce tornando exit status
            # 113 (attenzione: viene considerato fallimento!)
$ chmod 755 script.sh
$ ./script.sh
hello
0
./script.sh: lskdf: command not found
127
\end{verbatim}

\section{Strutture di controllo}
Come dicevamo prima, gli \emph{exit status} vengono utilizzati per le
espressioni condizionali che governano il programma:
\begin{verbatim}
if cmp a b > /dev/null   # ridirigiamo l'output su /dev/null
then echo "I file a e b sono identici."
else echo "I file a e b sono diversi."
fi
\end{verbatim}

Le condizioni sono prevalentemente controllate con il comando \texttt{test
\textit{expression}}, che torna 0 se l'espressione \`e vera, 1 altrimenti.
Molti sono gli argomenti di confronto che pu\`o accettare \texttt{test}, e
sarebbe inutile elencarli qui, quindi vi rimando alla manpage \textbf{test(1)}.
\`E utile comunque ricordare che gli operatori di confronto sono diversi per
interi e stringhe, e che esistono operatori di test per file e directory (ad
esempio, per sapere se un file esiste ed e` eseguibile).

La shell offre il comando built-in \verb_[ ]_ e le keyword
\verb_[[ ]]_ per valutare la condizione (usando il valore di
ritorno di \texttt{test}) al loro interno. La differenza tra i
due sta nel fatto che all'interno di \verb_[[ ]]_ non viene
effettuato filename expansion, inoltre operatori come \&\&, ||,
> e < vengono interpretati correttamente.

Ad esempio, per controllare l'esistenza di un argomento:
\begin{verbatim}
if [ -n "$1" ]
then
  lines=$1
fi
\end{verbatim}

Questo esempio ci introduce il cosiddetto \textit{blocco condizionale},
comunemente chiamato \textit{blocco if}.
Un blocco condizionale permette di eseguire comandi condizionati, ovvero di
scegliere, a seconda che una condizione sia vera o falsa, una lista di comandi
piuttosto che un'altra. La struttura di un blocco condizionale \`e la
seguente:
\begin{verbatim}
if list1
then
         list2
elif list3
then
         list4
else
         list5
fi
\end{verbatim}

Nel listato sopra, vengono innanzitutto eseguiti i comandi di \texttt{list1},
e se l'exit status dell'ultimo comando della lista \`e \texttt{0} (successo),
il blocco passa ad eseguire i comandi in \texttt{list2} e quindi esce; in
caso negativo (ovvero \texttt{list1} fallisce), vengono eseguiti i comandi in
\texttt{list3}, e ancora come prima, se la lista esce con successo vengono
eseguiti \texttt{list4} e quindi il blocco esce, altrimenti vengono eseguiti i
comandi in \texttt{list5}.

\`E importante notare che un blocco if pu\`o contenere zero o pi\`u sezioni
\texttt{elif}, e che la sezione \texttt{else} \`e sempre opzionale.

Un esempio un po' pi\`u lungo per capire meglio il funzionamento:
\begin{verbatim}
#!/bin/bash
stop=0
while [[ $stop -eq 0 ]]
do
        cat << ENDOFMENU
        1: sysadmin
        2: user
        3: guest
ENDOFMENU
echo "Your choice?"
read ch
if [[ "$ch" = "1" ]]; then
        sysadmin
elif [[ "$ch" = "2" ]]; then
        user
elif [[ "$ch" = "3" ]]; then
        guest
else
        echo error
fi
\end{verbatim}

\medskip
Per evitare molteplici utilizzi di \texttt{elif}, che possono risultare
scomodi e rendere meno leggibile il codice, un'altra struttura di controllo
\`e disponibile: il cosiddetto \texttt{case - in - esac}. La struttura \`e la
seguente:
\begin{verbatim}
case expression in
        value1)
                list1
                ;;
        value2)
                list2
                ;;
esac
\end{verbatim}
\texttt{expression} viene valutata, e il suo risultato viene confrontato con i
vari \texttt{value}, dal primo all'ultimo; quando il primo \texttt{value} che
corrisponde al risultato viene trovato, si esegue la \texttt{list}
corrispondente, e finita la sequenza di comandi si salta al \texttt{esac} del
blocco. \`E importante notare che i \texttt{value} possono contenere
wildcards, quindi mettere ad esempio \texttt{*$)$} come ultimo value ha senso
per prendere tutti i risultati che non rispecchiano i valori specificati
sopra.
Ad esempio:
\begin{verbatim}
#!/bin/bash
# Stampa gli utenti che consumano piu` spazio su HD
case "$1" in
  "")
    lines=50
    ;;
  *[!0-9]*)
    echo "Usage: `basename $0` usersnum";
    exit 1
    ;;
  *)
    lines=$1
    ;;
esac
du -s /home/* | sort -gr | head -$lines
\end{verbatim}

\medskip
Per ripetere pi\`u volte una lista di comandi finch\'e una data condizione non
sia vera, la shell ci mette a disposizione tre tipi di comandi:
\begin{itemize}
	\item \texttt{while - do - done}
	\item \texttt{until - do - done}
	\item \texttt{for - in - do - done}
\end{itemize}
Le strutture sono cos\`\i definite:
\begin{verbatim}
while list1
do
        list2
done

until list1
do
        list2
done
\end{verbatim}
per while e until, molto simili fra loro: nel while, \texttt{list1} viene
eseguita, e se l'ultimo comando esce con successo viene eseguita
\texttt{list2}, e si ritorna quindi all'inizio del ciclo con una nuova
esecuzione di \texttt{list1}, finch\'e quest'ultima non esce con status
fallimentare. Nell'until funziona alla stessa maniera, salvo il fatto che il
ciclo viene ripetuto finch\'e \texttt{list1} \`e falsa, e quando invece esce
con successo, il ciclo termina.

A questi due cicli si applicano due istruzioni: \texttt{break}, che causa
l'immediata uscita dal ciclo, e \texttt{continue}, che riporta immediatamente
il ciclo all'inizio.

Ad esempio, possiamo utilizzare un ciclo until e un while per stampare le
tabelline...
\begin{verbatim}
#!/bin/bash
if [ "$1" -eq "" ]; then
  echo "Usage: `basename $0` max"
  exit
fi
x=1
until [ $1 -gt $x ]
do
  y=1
  while [ $y -le $1 ]
  do
    echo `expr $x \* $y` ""
    y=`expr $y + 1`
  done
  echo
  x=`expr $x + 1`
done
\end{verbatim}

Il ciclo for, invece, ha la seguente struttura:
\begin{verbatim}
for name in words
do
        list
done
\end{verbatim}

Ciascuna parola nella lista \texttt{words} viene caricata nella variabile
\texttt{name} ad ogni iterazione del ciclo, e quindi viene eseguita
\texttt{list}. Se la clausola \texttt{in} viene omessa, al suo posto viene
utilizzata la variabile \texttt{\$*}.
Ancora un esempio:
\begin{verbatim}
#!/bin/bash
for color in red yellow blue green
do
  echo One color is $color
done
\end{verbatim}

\section{Comandi built-in}
\subsection{I/O}
\subsubsection*{\texttt{echo}}
Stampa i suoi argomenti su \texttt{stdout}. L'opzione \texttt{-n} viene
utilizzata per evitare che la stampa includa il newline finale. Utilizzato con
command substitution, pu\`o servire per settare una variabile:
\begin{verbatim}
a=`echo "HELLO" | tr A-Z a-z`
\end{verbatim}
Per abilitare l'interpretazione delle sequenze di escape, si specifica
l'opzione \texttt{-e}.

\subsubsection*{\texttt{printf}}
Versione migliorata di \texttt{echo}, la sintassi \`e \emph{simile} a quella
della funzione C \texttt{printf(3)}. L'interpretazione delle sequenze di
escape \`e abilitata di default.

\subsubsection*{\texttt{read \emph{var}}}
Legge l'input da stdin e lo archivia nella variabile specificata
(\texttt{var}). L'opzione \texttt{-p} evita la stampa dell'input,
\texttt{-nN} accetta solo N caratteri in input, e \texttt{-p} stampa una
stringa alla chiamata di \texttt{read}.
\begin{verbatim}
read -s -n1 -p "Premi un tasto..." keypress
echo; echo "Il tasto premuto e` \"$keypress\"."
\end{verbatim}

\subsection{variabili}
\subsubsection*{\texttt{declare}, \texttt{typeset}}
Permettono di restringere le propriet\`a di una variabile, una forma debole di
gestione dei tipi. L'opzione \texttt{-r} dichiara una variabile come
read-only (come \texttt{readonly}, \texttt{-i} come intera, \texttt{-a} come
array e \texttt{-x} esporta la variabile (come \texttt{export}).

\subsubsection*{\texttt{let}}
Esegue operazioni aritmetiche sulle variabili, in molti casi pu\`o funzionare
come una versione semplificata di \texttt{expr}.
\begin{verbatim}
let a=11       # assegna ad a il valore 11
let "a <<= 3"  # shift a sinistra di 3 posizioni
let "a /= 4"   # a viene diviso per 4 e il risultato viene riassegnato
let "a -= 5"   # sottrazione di 5 da a e riassegnamento
\end{verbatim}

\subsubsection*{\texttt{unset}}
Cancella il contenuto di una variabile

\subsection{Script execution}
\subsubsection*{\texttt{source}}
Conosciuto anche come \emph{dot command} (\texttt{.}). Esegue uno script senza
aprire una subshell. Corrisponde alla direttiva \verb_#include_ del
preprocessore C.
\begin{verbatim}
$ source ~/bash_profile
$ . ~/bash_profile
\end{verbatim}

\subsection{Comandi vari}
\subsubsection*{\texttt{true}, \texttt{false}}
Ritornano sempre rispettivamente 0 e 1.

\subsubsection*{\texttt{type $[$\emph{cmd}$]$}}
Stampa un messaggio che indica se \texttt{cmd} \`e una keyword, un comando
built-in oppure un comando esterno.

\subsubsection*{\texttt{shopt $[$\emph{options}$]$}}
Setta alcune opzioni della shell (ad esempio, \texttt{shopts -s cdspell}
permette il mispelling in \texttt{cd}).

\subsubsection*{\texttt{alias}, \texttt{unalias}}
Un \emph{alias} \`e una scorciatoia per abbreviare lunghe sequenze di comandi:
\begin{verbatim}
alias dir="ls -l"
alias rd="rm -r"
unalias dir
\end{verbatim}

\subsubsection*{\texttt{history}}
Visualizza l'elenco degli ultimi comandi eseguiti.

\subsection{Directory stack}
La shell Bash permette di manipolare una pila di directory con alcuni comandi
utili per visite ad albero.

\subsubsection*{\texttt{dirs}}
Stampa il contenuto del directory stack.

\subsubsection*{\texttt{pushd \emph{dirname}}}
Esegue il push della directory \emph{dirname} nello stack; successivamente si
sposta nella directory \emph{dirname} ed esegue \texttt{dirs}.

\subsubsection*{\texttt{popd}}
Esegue il pop dallo stack, si sposta sull'attuale top element ed esegue
\texttt{dirs}.

\subsubsection*{\texttt{\$DIRSTACK}}
Contiene il top element dello stack.

\section{File di configurazione}
\subsection{Inizializzazione}
La shell, all'avvio, fa il parsing di alcuni file di configurazione:

\subsubsection*{/etc/profile}
Settings sistem-wide validi per tutte le shell, non solo Bash.

\subsubsection*{/etc/bashrc}
Configurazione sistem-wide valida unicamente per Bash, contenente funzioni,
alias e modelli di comportamento.

\subsubsection*{\$HOME/.bash\_profile}
Settings specifici per l'utente relativi a Bash (se Bash non trova questo
file, cercher\`a il file \texttt{\$HOME/.profile}.

\subsubsection*{\$HOME/.bashrc}
Configurazione di Bash specifica per l'utente, contiene funzioni e alias.
Viene letto solo se la shell \`e di tipo interattivo o in esecuzione di
script.

\subsection{Terminazione}
\subsubsection*{\$HOME/.bash\_logout}
Al logout, Bash esegue lo script contenuto in questo file.

\section{Comandi esterni}



\appendix
% Appendix 1: GNU Free Documentation License
\input{A1_gnufdl.tex}

\end{document}

