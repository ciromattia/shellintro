\chapter{Programmazione della shell}

\section{Valutazione delle espressioni}
La shell non supporta direttamente (come faceva, per chi lo ricorda, il caro vecchio
\textbf{C$=$64}, avendo il \emph{BASIC} integrato) la valutazione delle
espressioni; a questo supplisce il comando \texttt{expr \textit{espressione}},
che ci ritorna il risultato della formula introdotta. Tutti i componenti
dell'espressione devono essere necessariamente separati da spazi tra di loro,
e tutti i metacaratteri devono essere ovviamente "escaped"\footnote{come
dicevamo prima, apponendo una \textit{backslash} ($\backslash$) al carattere
(ad esempio: $\backslash$*)}, altrimenti la shell li espander\`a secondo il
loro metasignificato.

Il risultato pu\`o essere numerico o una stringa, e pu\`o venire assegnato ad
una variabile con un uso opportuno del \emph{command substitution}.

Gli operatori disponibili sono:
\begin{itemize}
	\item Aritmetici:\texttt{\ \ \ \ + - * / \%}
	\item Confronto:\texttt{\ \ \ \ = != > < >= <=}
	\item Logici:\texttt{\ \ \ \ \ \ \ \ \& | !}
	\item Parentesi\footnote{devono essere prefissate dalla
	\emph{backslash}}: \texttt{\ \ \ ( )}
	\item Stringhe:
	\begin{itemize}
		\item \texttt{match \textit{string regexp}}
		\item \texttt{substr \textit{string start length}}
		\item \texttt{length \textit{string}}
		\item \texttt{index \textit{string charList}}
	\end{itemize}
\end{itemize}

\section{Exit status}
Ogni comando lanciato dalla shell ritorna un \emph{exit status}. Per
convenzione UNIX, un \emph{exit status} pari a 0 indica ``uscita con
successo'' (pari al valore \texttt{TRUE} nella valutazione delle espressioni
booleane), mentre un valore diverso indica ``uscita (spesso prematura) con
fallimento''. L'\emph{exit status} torna molto utile negli script per
controllare il flusso dell'esecuzione dei comandi, modificandolo in base ai
risutlati d'uscita prodotti. Proprio per questo, la shell ci offre il comando
\texttt{exit \textit{nn}} per far terminare il nostro script con \emph{exit
status} pari a \textit{nn}, e la variabile \emph{built-in} \texttt{\$?}, che
contiene l'\emph{exit status} dell'ultimo comando eseguito:
\begin{verbatim}
$ cat script.sh
#!/bin/bash
echo hello
echo $?     # "echo hello" ha tornato exit status 0 (successo)
lskdf       # comando inesistente
echo $?     # "lskdf" torna exit status diverso da 0 (fallimento)
exit 113    # il nostro script esce tornando exit status
            # 113 (attenzione: viene considerato fallimento!)
$ chmod 755 script.sh
$ ./script.sh
hello
0
./script.sh: lskdf: command not found
127
\end{verbatim}

\section{Strutture di controllo}
Come dicevamo prima, gli \emph{exit status} vengono utilizzati per le
espressioni condizionali che governano il programma:
\begin{verbatim}
if cmp a b > /dev/null   # ridirigiamo l'output su /dev/null
then echo "I file a e b sono identici."
else echo "I file a e b sono diversi."
fi
\end{verbatim}

Le condizioni sono prevalentemente controllate con il comando \texttt{test
\textit{expression}}, che torna 0 se l'espressione \`e vera, 1 altrimenti.
Molti sono gli argomenti di confronto che pu\`o accettare \texttt{test}, e
sarebbe inutile elencarli qui, quindi vi rimando alla manpage \textbf{test(1)}.
\`E utile comunque ricordare che gli operatori di confronto sono diversi per
interi e stringhe, e che esistono operatori di test per file e directory (ad
esempio, per sapere se un file esiste ed e` eseguibile).

La shell offre il comando built-in \verb_[ ]_ e le keyword
\verb_[[ ]]_ per valutare la condizione (usando il valore di
ritorno di \texttt{test}) al loro interno. La differenza tra i
due sta nel fatto che all'interno di \verb_[[ ]]_ non viene
effettuato filename expansion, inoltre operatori come \&\&, ||,
> e < vengono interpretati correttamente.

Ad esempio, per controllare l'esistenza di un argomento:
\begin{verbatim}
if [ -n "$1" ]
then
  lines=$1
fi
\end{verbatim}

Questo esempio ci introduce il cosiddetto \textit{blocco condizionale},
comunemente chiamato \textit{blocco if}.
Un blocco condizionale permette di eseguire comandi condizionati, ovvero di
scegliere, a seconda che una condizione sia vera o falsa, una lista di comandi
piuttosto che un'altra. La struttura di un blocco condizionale \`e la
seguente:
\begin{verbatim}
if list1
then
         list2
elif list3
then
         list4
else
         list5
fi
\end{verbatim}

Nel listato sopra, vengono innanzitutto eseguiti i comandi di \texttt{list1},
e se l'exit status dell'ultimo comando della lista \`e \texttt{0} (successo),
il blocco passa ad eseguire i comandi in \texttt{list2} e quindi esce; in
caso negativo (ovvero \texttt{list1} fallisce), vengono eseguiti i comandi in
\texttt{list3}, e ancora come prima, se la lista esce con successo vengono
eseguiti \texttt{list4} e quindi il blocco esce, altrimenti vengono eseguiti i
comandi in \texttt{list5}.

\`E importante notare che un blocco if pu\`o contenere zero o pi\`u sezioni
\texttt{elif}, e che la sezione \texttt{else} \`e sempre opzionale.

Un esempio un po' pi\`u lungo per capire meglio il funzionamento:
\begin{verbatim}
#!/bin/bash
stop=0
while [[ $stop -eq 0 ]]
do
        cat << ENDOFMENU
        1: sysadmin
        2: user
        3: guest
ENDOFMENU
echo "Your choice?"
read ch
if [[ "$ch" = "1" ]]; then
        sysadmin
elif [[ "$ch" = "2" ]]; then
        user
elif [[ "$ch" = "3" ]]; then
        guest
else
        echo error
fi
\end{verbatim}

\medskip
Per evitare molteplici utilizzi di \texttt{elif}, che possono risultare
scomodi e rendere meno leggibile il codice, un'altra struttura di controllo
\`e disponibile: il cosiddetto \texttt{case - in - esac}. La struttura \`e la
seguente:
\begin{verbatim}
case expression in
        value1)
                list1
                ;;
        value2)
                list2
                ;;
esac
\end{verbatim}
\texttt{expression} viene valutata, e il suo risultato viene confrontato con i
vari \texttt{value}, dal primo all'ultimo; quando il primo \texttt{value} che
corrisponde al risultato viene trovato, si esegue la \texttt{list}
corrispondente, e finita la sequenza di comandi si salta al \texttt{esac} del
blocco. \`E importante notare che i \texttt{value} possono contenere
wildcards, quindi mettere ad esempio \texttt{*$)$} come ultimo value ha senso
per prendere tutti i risultati che non rispecchiano i valori specificati
sopra.
Ad esempio:
\begin{verbatim}
#!/bin/bash
# Stampa gli utenti che consumano piu` spazio su HD
case "$1" in
  "")
    lines=50
    ;;
  *[!0-9]*)
    echo "Usage: `basename $0` usersnum";
    exit 1
    ;;
  *)
    lines=$1
    ;;
esac
du -s /home/* | sort -gr | head -$lines
\end{verbatim}

\medskip
Per ripetere pi\`u volte una lista di comandi finch\'e una data condizione non
sia vera, la shell ci mette a disposizione tre tipi di comandi:
\begin{itemize}
	\item \texttt{while - do - done}
	\item \texttt{until - do - done}
	\item \texttt{for - in - do - done}
\end{itemize}
Le strutture sono cos\`\i definite:
\begin{verbatim}
while list1
do
        list2
done

until list1
do
        list2
done
\end{verbatim}
per while e until, molto simili fra loro: nel while, \texttt{list1} viene
eseguita, e se l'ultimo comando esce con successo viene eseguita
\texttt{list2}, e si ritorna quindi all'inizio del ciclo con una nuova
esecuzione di \texttt{list1}, finch\'e quest'ultima non esce con status
fallimentare. Nell'until funziona alla stessa maniera, salvo il fatto che il
ciclo viene ripetuto finch\'e \texttt{list1} \`e falsa, e quando invece esce
con successo, il ciclo termina.

A questi due cicli si applicano due istruzioni: \texttt{break}, che causa
l'immediata uscita dal ciclo, e \texttt{continue}, che riporta immediatamente
il ciclo all'inizio.

Ad esempio, possiamo utilizzare un ciclo until e un while per stampare le
tabelline...
\begin{verbatim}
#!/bin/bash
if [ "$1" -eq "" ]; then
  echo "Usage: `basename $0` max"
  exit
fi
x=1
until [ $1 -gt $x ]
do
  y=1
  while [ $y -le $1 ]
  do
    echo `expr $x \* $y` ""
    y=`expr $y + 1`
  done
  echo
  x=`expr $x + 1`
done
\end{verbatim}

Il ciclo for, invece, ha la seguente struttura:
\begin{verbatim}
for name in words
do
        list
done
\end{verbatim}

Ciascuna parola nella lista \texttt{words} viene caricata nella variabile
\texttt{name} ad ogni iterazione del ciclo, e quindi viene eseguita
\texttt{list}. Se la clausola \texttt{in} viene omessa, al suo posto viene
utilizzata la variabile \texttt{\$*}.
Ancora un esempio:
\begin{verbatim}
#!/bin/bash
for color in red yellow blue green
do
  echo One color is $color
done
\end{verbatim}

\section{Comandi built-in}
\subsection{I/O}
\subsubsection*{\texttt{echo}}
Stampa i suoi argomenti su \texttt{stdout}. L'opzione \texttt{-n} viene
utilizzata per evitare che la stampa includa il newline finale. Utilizzato con
command substitution, pu\`o servire per settare una variabile:
\begin{verbatim}
a=`echo "HELLO" | tr A-Z a-z`
\end{verbatim}
Per abilitare l'interpretazione delle sequenze di escape, si specifica
l'opzione \texttt{-e}.

\subsubsection*{\texttt{printf}}
Versione migliorata di \texttt{echo}, la sintassi \`e \emph{simile} a quella
della funzione C \texttt{printf(3)}. L'interpretazione delle sequenze di
escape \`e abilitata di default.

\subsubsection*{\texttt{read \emph{var}}}
Legge l'input da stdin e lo archivia nella variabile specificata
(\texttt{var}). L'opzione \texttt{-p} evita la stampa dell'input,
\texttt{-nN} accetta solo N caratteri in input, e \texttt{-p} stampa una
stringa alla chiamata di \texttt{read}.
\begin{verbatim}
read -s -n1 -p "Premi un tasto..." keypress
echo; echo "Il tasto premuto e` \"$keypress\"."
\end{verbatim}

\subsection{variabili}
\subsubsection*{\texttt{declare}, \texttt{typeset}}
Permettono di restringere le propriet\`a di una variabile, una forma debole di
gestione dei tipi. L'opzione \texttt{-r} dichiara una variabile come
read-only (come \texttt{readonly}, \texttt{-i} come intera, \texttt{-a} come
array e \texttt{-x} esporta la variabile (come \texttt{export}).

\subsubsection*{\texttt{let}}
Esegue operazioni aritmetiche sulle variabili, in molti casi pu\`o funzionare
come una versione semplificata di \texttt{expr}.
\begin{verbatim}
let a=11       # assegna ad a il valore 11
let "a <<= 3"  # shift a sinistra di 3 posizioni
let "a /= 4"   # a viene diviso per 4 e il risultato viene riassegnato
let "a -= 5"   # sottrazione di 5 da a e riassegnamento
\end{verbatim}

\subsubsection*{\texttt{unset}}
Cancella il contenuto di una variabile

\subsection{Script execution}
\subsubsection*{\texttt{source}}
Conosciuto anche come \emph{dot command} (\texttt{.}). Esegue uno script senza
aprire una subshell. Corrisponde alla direttiva \verb_#include_ del
preprocessore C.
\begin{verbatim}
$ source ~/bash_profile
$ . ~/bash_profile
\end{verbatim}

\subsection{Comandi vari}
\subsubsection*{\texttt{true}, \texttt{false}}
Ritornano sempre rispettivamente 0 e 1.

\subsubsection*{\texttt{type $[$\emph{cmd}$]$}}
Stampa un messaggio che indica se \texttt{cmd} \`e una keyword, un comando
built-in oppure un comando esterno.

\subsubsection*{\texttt{shopt $[$\emph{options}$]$}}
Setta alcune opzioni della shell (ad esempio, \texttt{shopts -s cdspell}
permette il mispelling in \texttt{cd}).

\subsubsection*{\texttt{alias}, \texttt{unalias}}
Un \emph{alias} \`e una scorciatoia per abbreviare lunghe sequenze di comandi:
\begin{verbatim}
alias dir="ls -l"
alias rd="rm -r"
unalias dir
\end{verbatim}

\subsubsection*{\texttt{history}}
Visualizza l'elenco degli ultimi comandi eseguiti.

\subsection{Directory stack}
La shell Bash permette di manipolare una pila di directory con alcuni comandi
utili per visite ad albero.

\subsubsection*{\texttt{dirs}}
Stampa il contenuto del directory stack.

\subsubsection*{\texttt{pushd \emph{dirname}}}
Esegue il push della directory \emph{dirname} nello stack; successivamente si
sposta nella directory \emph{dirname} ed esegue \texttt{dirs}.

\subsubsection*{\texttt{popd}}
Esegue il pop dallo stack, si sposta sull'attuale top element ed esegue
\texttt{dirs}.

\subsubsection*{\texttt{\$DIRSTACK}}
Contiene il top element dello stack.

\section{File di configurazione}
\subsection{Inizializzazione}
La shell, all'avvio, fa il parsing di alcuni file di configurazione:

\subsubsection*{/etc/profile}
Settings sistem-wide validi per tutte le shell, non solo Bash.

\subsubsection*{/etc/bashrc}
Configurazione sistem-wide valida unicamente per Bash, contenente funzioni,
alias e modelli di comportamento.

\subsubsection*{\$HOME/.bash\_profile}
Settings specifici per l'utente relativi a Bash (se Bash non trova questo
file, cercher\`a il file \texttt{\$HOME/.profile}.

\subsubsection*{\$HOME/.bashrc}
Configurazione di Bash specifica per l'utente, contiene funzioni e alias.
Viene letto solo se la shell \`e di tipo interattivo o in esecuzione di
script.

\subsection{Terminazione}
\subsubsection*{\$HOME/.bash\_logout}
Al logout, Bash esegue lo script contenuto in questo file.

\section{Comandi esterni}

